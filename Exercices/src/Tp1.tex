\documentclass[11pt, a4paper]{article} 

%\parindent 5mm 
%\parskip 0mm 
%\addtolength{\textwidth}{3 truecm} 
%\addtolength{\textheight}{1 truecm} 
%\setlength{\voffset}{-.6 truecm} 
%\setlength{\hoffset}{-1.3 truecm} 

\usepackage[lmargin=25mm,rmargin=25mm,tmargin=20mm,bmargin=20mm]{geometry}

\usepackage[frenchb]{babel} 
\usepackage{amsthm, amssymb, amsmath} 
\usepackage[T1]{fontenc} 
\usepackage{tikz, pgf} 
\usepackage{graphicx} 
\usepackage{xcolor} 
\usepackage{url, enumerate} 
\usepackage{epsfig}

\begin{document} 

\title{MATH-F-307: S\'eance d'exercices 1} 
\author{} 
\date{25 septembre 2015}

\theoremstyle{plain} 
\newtheorem*{theo}{Th\'eor\`eme}
%\newtheorem{defi}[theo]{D\'efinition} 
%\newtheorem{lemme}[theo]{Lemme} 
%\newtheorem{corol}[theo]{Corollaire} 
%\newtheorem{prop}[theo]{Proposition} 
%\newtheorem{fait}[theo]{Fait} 
%\newtheorem{aff}{Affirmation} 
%\newtheorem*{rapp}{Rappel} 
%\newtheorem{conj}[theo]{Conjecture} 

\theoremstyle{definition} 
%\newtheorem{probl}[theo]{Probl\`eme ouvert} 
%\newtheorem*{ex}{Exemple} 
%\newtheorem*{rem}{Remarque} 
\newtheorem{exo}{Exercice}

\newcommand{\R}{\mathbb{R}} 
\newcommand{\Z}{\mathbb{Z}} 
\newcommand{\N}{\mathbb{N}} 

%\renewcommand{\theenumi}{(\alph{enumi})}
%\renewcommand{\labelenumi}{\theenumi}
\renewcommand{\FrenchLabelItem}{\textbullet}

\maketitle


\begin{exo}
Construisez un graphe simple et connexe sur $8$ sommets tel que chaque sommet est contenu dans exactement trois ar\^etes. Pouvez-vous faire la m\^eme chose avec $9$ sommets?
\end{exo}

\begin{exo}
Dans un groupe de personnes, il y a toujours deux individus qui connaissent exactement le m\^eme nombre de membres du groupe.
\begin{enumerate}
\item Formalisez cette propri\'et\'e dans le vocabulaire des graphes.
\item D\'emontrez cette propri\'et\'e (par l'absurde).
\end{enumerate}
\end{exo}

\begin{exo}
Soit $n\geq 2$ et soit $G$ un graphe simple avec $2n$ sommets et $n^2+1$ ar\^etes. Montrez que $G$ contient un triangle.
\end{exo}

\begin{exo}
Soit $G$ un graphe simple avec $2p$ sommets. On suppose que le degr\'e de chaque sommet est au moins \'egal \`a $p$. D\'emontrez que ce graphe est connexe.
\end{exo}

\begin{exo}
Soit $G$ un graphe simple.
\begin{enumerate}
\item On suppose que $G$ est connexe et que $x$ est un sommet de $G$ de degr\'e $1$. Prouvez que $G\setminus\{x\}$ est connexe.
\item D\'eduisez-en que, si $G$ est connexe et $|V(G)|=n\geq 2$, alors $G$ contient au moins $n-1$ ar\^etes.
\end{enumerate}
\end{exo}

\begin{exo}
Donnez un graphe simple et connexe sur au moins $5$ sommets qui est:
\begin{itemize}
\item hamiltonien et eul\'erien;
\item hamiltonien et non eul\'erien;
\item non hamiltonien et eul\'erien;
\item non hamiltonien et non eul\'erien.
\end{itemize}
\end{exo}

\begin{exo}
Le graphe

\begin{center}
\scalebox{.825}{
\begin{tikzpicture}

\tikzstyle{every node}=[circle, draw, fill=white, inner sep=0pt, minimum size=5pt]

\node (a00) at (0,0) {};
\node (aleft) at (-1,0) {};
\node (aabove) at (0,1) {};
\node (adown) at (0,-1) {};
\node (aright) at (1,0) {};
\node (aright2) at (2,0) {};

\draw (a00) -- (aleft);
\draw (a00) -- (aright);
\draw (a00) -- (adown);
\draw (a00) -- (aabove);
\draw (aright) -- (aright2);
\end{tikzpicture}}
\end{center}

est-il isomorphe \`a un (ou \`a plusieurs) des graphes ci-dessous~?

\begin{center}
\scalebox{.825}{
\begin{minipage}[t]{0.2\linewidth}
   \begin{tikzpicture}
		\tikzstyle{every node}=[circle, draw, fill=white, inner sep=0pt, minimum size=5pt]

        \node (a00) at (0,0) {};
        \node (leftup) at (-1,1) {};
        \node (leftdown) at (-1,-1) {};
        \node (a01) at (1,0) {};
        \node (rightup) at (2,1) {};
        \node (rightdown) at (2,-1) {};

		\draw (a00) -- (leftup);
		\draw (a00) -- (leftdown);
		\draw (a00) -- (a01);
		\draw (a01) -- (rightup);
		\draw (a01) -- (rightdown);
	\end{tikzpicture}
\end{minipage}

\begin{minipage}[t]{0.2\linewidth}
   \begin{tikzpicture}
		\tikzstyle{every node}=[circle, draw, fill=white, inner sep=0pt, minimum size=5pt]

        \node (a0) at (0,0) {};
        \node (a1) at (1,0) {};
        \node (a0side) at (-1,0) {};
        \node (a0up) at (0,1) {};
        \node (a0down) at (0,-1) {};
        \node (a1side) at (2,0) {};
        \node (a1up) at (1,1) {};
        \node (a1down) at (1,-1) {};

		\draw (a0side) -- (a0) -- (a1) -- (a1side);
		\draw (a0up) -- (a0) -- (a0down);
		\draw (a1up) -- (a1) -- (a1down);

	\end{tikzpicture}
\end{minipage}

\begin{minipage}[t]{0.2\linewidth}
   \begin{tikzpicture}
		\tikzstyle{every node}=[circle, draw, fill=white, inner sep=0pt, minimum size=5pt]

        \node (a00) at (0,0) {};
		\node (a0up) at (0,1) {};
		\node (a0down) at (0,-1) {};
		\node (a0ddown) at (0,-2) {};
		\node (a0left) at (-1,1) {};
		\node (a0right) at (1,1) {};

		\draw (a0up) -- (a00) -- (a0down) -- (a0ddown);
		\draw (a00) -- (a0left);
		\draw (a00) -- (a0right);
	\end{tikzpicture}
\end{minipage}

\begin{minipage}[t]{0.2\linewidth}
	\hspace{-1cm}
   \begin{tikzpicture}
		\tikzstyle{every node}=[circle, draw, fill=white, inner sep=0pt, minimum size=5pt]

        \node (a00) at (0,0) {};
        \node (a0up) at (0,1) {};
        \node (a0down) at (0,-1) {};
        \node (a0left) at (-1,0) {};
        \node (a0right) at (1,0) {};
        \node (a0rright) at (2,0) {};

		\draw (a0left) -- (a00) -- (a0right) -- (a0rright);
		\draw (a0up) -- (a00) -- (a0down);
		\draw (a0up) -- (a0left) -- (a0down);
	\end{tikzpicture}
\end{minipage}

\begin{minipage}[t]{0.2\linewidth}
   \begin{tikzpicture}
		\tikzstyle{every node}=[circle, draw, fill=white, inner sep=0pt, minimum size=5pt]

        \node (a00) at (0,0) {};
		\node (a0up) at (0,1) {};
		\node (a0down) at (0,-1) {};
		\node (a0ddown) at (0,-2) {};
		\node (a0left) at (-1,-1) {};
		\node (a0right) at (1,0) {};

		\draw (a0up) -- (a00) -- (a0down) -- (a0ddown);
		\draw (a00) -- (a0left);
		\draw (a0down) -- (a0right);
	\end{tikzpicture}
\end{minipage}

\begin{minipage}[t]{0.2\linewidth}
   \begin{tikzpicture}
		\tikzstyle{every node}=[circle, draw, fill=white, inner sep=0pt, minimum size=5pt]

        \node (a00) at (0,0) {};
		\node (a0left) at (-1,0) {};
		\node (a0right) at (1,0) {};
		\node (a0rright) at (2,0) {};
		\node (a0upleft) at (0,1) {};
		\node (a0upright) at (2,1) {};

		\draw (a0left) -- (a00) -- (a0right) -- (a0rright);
		\draw (a0upleft) -- (a0right) -- (a0upright);
	\end{tikzpicture}
\end{minipage}}
\end{center}
\end{exo}

\newpage
\begin{exo}
Les graphes suivants sont-ils isomorphes~? (Ne vous contentez pas d'une justification approximative : essayez de d\'emontrer rigoureusement vos affirmations.)

\begin{figure}[h!]
\centering

\begin{minipage}[t]{0.2\linewidth}
	\begin{tikzpicture}[scale=0.4]
		\tikzstyle{every node}=[circle, draw, fill=white, inner sep=0pt, minimum size=5pt]

		\node (v6) at (-3.5,2.5) {};
		\node (v5) at (-5.5,2.5) {};
		\node (v8) at (-5.5,0.5) {};
		\node (v7) at (-3.5,0.5) {};
		\node (v1) at (-7,4) {};
		\node (v2) at (-2,4) {};
		\node (v4) at (-7,-1) {};
		\node (v3) at (-2,-1) {};
		\draw  (v1) edge (v2);
		\draw  (v3) edge (v2);
		\draw  (v4) edge (v3);
		\draw  (v1) edge (v4);
		\draw  (v5) edge (v6);
		\draw  (v6) edge (v7);
		\draw  (v7) edge (v8);
		\draw  (v8) edge (v5);
		\draw  (v5) edge (v1);
		\draw  (v6) edge (v2);
	\end{tikzpicture}
\end{minipage}
\hspace{1.5cm}
\begin{minipage}[t]{0.2\linewidth}
	\begin{tikzpicture}[scale=0.4]
		\tikzstyle{every node}=[circle, draw, fill=white, inner sep=0pt, minimum size=5pt]

		\node (v6) at (-3.5,2.5) {};
		\node (v5) at (-5.5,2.5) {};
		\node (v8) at (-5.5,0.5) {};
		\node (v7) at (-3.5,0.5) {};
		\node (v1) at (-7,4) {};
		\node (v2) at (-2,4) {};
		\node (v4) at (-7,-1) {};
		\node (v3) at (-2,-1) {};
		\draw  (v1) edge (v2);
		\draw  (v3) edge (v2);
		\draw  (v4) edge (v3);
		\draw  (v1) edge (v4);
		\draw  (v5) edge (v6);
		\draw  (v6) edge (v7);
		\draw  (v7) edge (v8);
		\draw  (v8) edge (v5);
		\draw  (v5) edge (v1);
		\draw  (v7) edge (v3);
	\end{tikzpicture}
\end{minipage}
\caption{}
\end{figure}%

\begin{figure}[h!]
\centering

\begin{minipage}[t]{0.2\linewidth}
   \begin{tikzpicture}[scale=0.6]
		\tikzstyle{every node}=[circle, draw, fill=white, inner sep=0pt, minimum size=5pt]

        \node (v6) at (0,0) {};
		\node (v7) at (1.5,0) {};
		\node (v5) at (0,-1.5) {};
		\node (v1) at (-1,1) {};
		\node (v2) at (2.5,1) {};
		\node (v4) at (-1,-2.5) {};
		\node (v3) at (2.5,-2.5) {};
		\draw (v1) -- (v2) -- (v2) -- (v3) -- (v3) -- (v4) -- (v4) -- (-1,1);
		\draw (v5) -- (v6) -- (v6) -- (v7) -- (v7) -- (v3) -- (v3) -- (v5) -- (v6) -- (v1);
		\draw (v2) -- (v7);
	\end{tikzpicture}
\end{minipage}
\hspace{1.5cm}
\begin{minipage}[t]{0.2\linewidth}
   \begin{tikzpicture}[scale=0.6];
		\tikzstyle{every node}=[circle, draw, fill=white, inner sep=0pt, minimum size=5pt]

        \node (v1) at (-0.5,0.25) {};
		\node (v2) at (-1.5,2) {};
		\node (v5) at (-1.5,-1.5) {};
		\node (v6) at (1,1) {};
		\node (v3) at (2,2) {};
		\node (v7) at (1,-0.5) {};
		\node (v4) at (2,-1.5) {};
		\draw (v2) -- (v3) -- (v4) -- (v5) -- (v2) -- (v1) -- (v6) -- (v7) -- (v1) -- (v5);
		\draw (v3) -- (v6) -- (v7) -- (v4);

	\end{tikzpicture}
\end{minipage}
\caption{}
\end{figure}

\begin{figure}[h!]
\centering
\begin{minipage}[t]{0.2\linewidth}
   \begin{tikzpicture}[scale=2]
   		\node[draw=none,minimum size=3cm,regular polygon,regular polygon sides=7] (a) {};

		\foreach \x in {1,2,...,7}
			\draw[fill=white,inner sep=0pt, minimum size=5pt] (a.corner \x) circle (1.5pt);
		  
		\draw (a.corner 1) -- (a.corner 2) -- (a.corner 3) -- (a.corner 4) -- (a.corner 5) -- (a.corner 6) -- (a.corner 7) -- (a.corner 1);
		\draw (a.corner 2) -- (a.corner 7) -- (a.corner 5) -- (a.corner 3) -- (a.corner 1) -- (a.corner 6) -- (a.corner 4) -- (a.corner 2) ;

	\end{tikzpicture}
\end{minipage}
\hspace{1.5cm}
\begin{minipage}[t]{0.2\linewidth}
   \begin{tikzpicture}[scale=2]
		\node[draw=none,minimum size=3cm,regular polygon,regular polygon sides=7] (a) {};

		\foreach \x in {1,2,...,7}
  			\draw[fill=white,inner sep=0pt, minimum size=5pt] (a.corner \x) circle (1.5pt);

		\draw (a.corner 1) -- (a.corner 5) -- (a.corner 2) -- (a.corner 6) -- (a.corner 3) -- (a.corner 7) -- (a.corner 4) -- (a.corner 1);
		\draw (a.corner 1) -- (a.corner 2) -- (a.corner 3) -- (a.corner 4) -- (a.corner 5) -- (a.corner 6) -- (a.corner 7) -- (a.corner 1);


	\end{tikzpicture}
\end{minipage}
\caption{}
\end{figure}

\end{exo}


\end{document}