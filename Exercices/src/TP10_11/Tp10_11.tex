
\section{Séance 10 et 11}

\begin{exo}
Que vaut
$$
\sum_{n=0}^\infty H_n \frac{1}{10^n}\quad ?
$$
(Rappel~: $H_n$ est le $n$-\`eme nombre harmonique.)
\end{exo}

Nous savons que $$ C(x) = \sum_{n=0}^\infty H_n x^n = \frac{1}{1-x} ln(\frac{1}{1-x})$$

Donc $$ \sum_{n=0}^\infty H_n \frac{1}{10^n} = \frac{1}{1-\frac{1}{10}} ln(\frac{1}{1-\frac{1}{10}}) = \frac{10}{9} ln(\frac{10}{9}) = \frac{10}{9} (ln(10) - ln(9)) $$
%--------------------

\begin{exo}
Trouver la fonction g\'en\'eratrice ordinaire de $(2^n + 3^n)_{n \in \mathbb{N}}$, en forme close.
\end{exo}

%--------------------

\begin{exo} (Examen janvier 2011.) 
Calculer la somme de chacune des s\'eries suivantes.
%
\begin{enumerate}[a)]
\item $\displaystyle \sum_{n=0}^\infty \frac{H_n}{2^n}$
\item $\displaystyle \sum_{n=0}^\infty {n \choose 2} \frac{1}{10^n}$
\end{enumerate}
\end{exo}

\begin{enumerate}[a)]
\item Il s'agit de la FGO de $\frac{1}{1-\frac{1}{2}}*ln(\frac{1}{1-\frac{1}{2}}) = 2*ln(2)$.

\item Nous savons que 
\end{enumerate}

%--------------------

\begin{exo} (Examen ao\^ut 2011.) 
Calculer la somme de chacune des s\'eries suivantes.
%
\begin{enumerate}[a)]
\item $\displaystyle \sum_{n=1}^\infty \frac{1}{2^n}$
\item $\displaystyle \sum_{n=1}^\infty \frac{n}{2^n}$
\item $\displaystyle \sum_{n=1}^\infty \frac{1}{n 2^n}$ 
\end{enumerate}
\end{exo}

ATTENTION: n=1 dans chaque exercice!

\begin{enumerate}[a)]
\item $\displaystyle \sum_{n=1}^\infty \frac{1}{2^n} = \sum_{n=0}^\infty \frac{1}{2^n} - 1 = \frac{1}{1-\frac{1}{2}} - 1 = 2 - 1 = 1 $
\item $\displaystyle \sum_{n=1}^\infty \frac{n}{2^n}$
\item $\displaystyle \sum_{n=1}^\infty \frac{1}{n 2^n}$ 
\end{enumerate}

%--------------------

\begin{exo}
Quelle est la FGO de $(1,1+3,1+3+3^2,1+3+3^2+3^3,\ldots)$?
\end{exo}

%--------------------

\begin{exo} (Examen septembre 2015.)
R\'esolvez par la m\'ethode des fonctions g\'en\'eratrices l'\'equation de r\'ecurrence 
$$
a_n-3a_{n-1}=4^n \quad \mathrm{avec} \quad a_0=1.
$$
\end{exo}

%--------------------

\begin{exo}
Un collectionneur excentrique rafolle des pavages de rectangles $2 \times n$ par des dominos verticaux $2 \times 1$ et horizontaux $1 \times 2$. Il paye sans h\'esiter $4$\euro{} par domino vertical et $1$\euro{} par domino horizontal. Pour combien de pavages sera-t-il pr\^et \`a payer $n$\euro~?
\end{exo}

%--------------------