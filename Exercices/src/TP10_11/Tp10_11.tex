
\section{Séance 10 et 11}

\begin{exo}
Que vaut
\[
\sum_{n=0}^\infty H_n \frac{1}{10^n}\quad ?
\]
(Rappel~: $H_n$ est le $n$-\`eme nombre harmonique.)
\end{exo}

Nous savons que \[ C(x) = \sum_{n=0}^\infty H_n x^n = \frac{1}{1-x} ln(\frac{1}{1-x}) \]

Donc \[ \sum_{n=0}^\infty H_n \frac{1}{10^n} = \frac{1}{1-\frac{1}{10}} ln(\frac{1}{1-\frac{1}{10}}) = \frac{10}{9} ln(\frac{10}{9}) = \frac{10}{9} (ln(10) - ln(9)) \]
%--------------------

\begin{exo}
Trouver la fonction g\'en\'eratrice ordinaire de $(2^n + 3^n)_{n \in \mathbb{N}}$, en forme close.
\end{exo}

Posons:

\[
A(x) = \sum_{n=0}^{\infty} 2^n x^n = \sum_{n=0}^{\infty} (2x)^n = \frac{1}{1-2x}
\]

\[
B(x) = \sum_{n=0}^{\infty} 3^n x^n = \sum_{n=0}^{\infty} (3x)^n = \frac{1}{1-3x}
\]

Donc:

\[
\sum_{n=0}^{\infty} (2^n + 3^n) \cdot x^n = A(x) + B(x) = \frac{1}{1-2x} + \frac{1}{1-3x}
\]

%--------------------

\begin{exo} (Examen janvier 2011.) 
Calculer la somme de chacune des s\'eries suivantes.
%
\begin{enumerate}[a)]
\item $\displaystyle \sum_{n=0}^\infty \frac{H_n}{2^n}$
\item $\displaystyle \sum_{n=0}^\infty {n \choose 2} \frac{1}{10^n}$
\end{enumerate}
\end{exo}

\begin{enumerate}[a)]
\item Il s'agit de la FGO de \[ \frac{1}{1-\frac{1}{2}}*ln(\frac{1}{1-\frac{1}{2}}) = 2*ln(2) \]

\item Nous savons que la FGO de $\left( {n \choose k} \right)_{n \in \N}, \;$ k fixé est: \[ \frac{x^k}{(1-x)^{k+1}} \]

Donc cette série vaut, pour $k=2$ et $x = \frac{1}{10}$ 
\[
\frac{ (\frac{1}{10})^2  }{ (1-\frac{1}{10})^{2-1} } = \frac{ \frac{1}{100} }{ \frac{9^3}{1000}} = \frac{10}{729}
\]
\end{enumerate}

\newpage

%--------------------

\begin{exo} (Examen ao\^ut 2011.) 
Calculer la somme de chacune des s\'eries suivantes.
%
\begin{enumerate}[a)]
\item $\displaystyle \sum_{n=1}^\infty \frac{1}{2^n}$
\item $\displaystyle \sum_{n=1}^\infty \frac{n}{2^n}$
\item $\displaystyle \sum_{n=1}^\infty \frac{1}{n 2^n}$ 
\end{enumerate}
\end{exo}

ATTENTION: n=1 dans chaque exercice!

\begin{enumerate}[a)]
\item $\displaystyle \sum_{n=1}^\infty \frac{1}{2^n} = -1 + \sum_{n=0}^\infty \frac{1}{2^n}= -1 + \frac{1}{1-\frac{1}{2}} = -1 + 2 = 1 $
\item $\displaystyle \sum_{n=1}^\infty \frac{n}{2^n} = \frac{1/2}{ (1-1/2)^2 } = \frac{1/2}{(1/2)^2} = \frac{1}{1/2} = 2$
\item $\displaystyle \sum_{n=1}^\infty \frac{1}{n 2^n} = ??$ 
\end{enumerate}

%--------------------

\begin{exo}
Quelle est la FGO de $(1,1+3,1+3+3^2,1+3+3^2+3^3,\ldots)$?
\end{exo}

Posons la suite $a_n = \sum_{i=0}^n z^i$ pour $n\in \N$.

La FGO de la suite $(a_n)_{n\in \N}$ est:

\begin{align*}
\sum_{n=0}^{\infty} ( \sum_{k=0}^{\infty} z^k \cdot 1) x^n &= (\sum_{n=0}^{\infty} z^n x^n) (\sum_{n=0}^{\infty} 1 x^n) \\
&= (\sum_{n=0}^{\infty} (zx)^n) (\sum_{n=0}^{\infty} x^n)\\
&= \frac{1}{1-3x} \cdot \frac{1}{1-x}\\
&= \frac{1}{(1-3x)(1-x)}
\end{align*}

\newpage
%--------------------

\begin{exo} (Examen septembre 2015.)
R\'esolvez par la m\'ethode des fonctions g\'en\'eratrices l'\'equation de r\'ecurrence 
\[
a_n-3a_{n-1}=4^n \quad \mathrm{avec} \quad a_0=1.
\]
\end{exo}

Soit A(x) = $\sum_{n=0}^{\infty} a_n x^n \;$ la FGO de $(a_n)_{n\in \N}$

Nous avons:

\begin{align*}
A(x) &= \sum_{n=0}^{\infty} a_n x^n \\
	&= a_0 + \sum_{n=1}^{\infty} a_n x^n \qquad,a_0 = 1 \qquad a_n = 4^n + 3a_{n-1} \\
	&= 1 + \sum_{n=1}^{\infty} (4^n + 3a_{n-1}) x^n \\
	&= 1 + \sum_{n=1}^{\infty} 4^n x^n + \sum_{n=1}^{\infty} 3a_{n-1} x^n \\
	&= \sum_{n=0}^{\infty} (4x)^n + 3x \cdot \underbrace{\sum_{n=0}^{\infty} a_n x^n}_{=A(x)}
\end{align*}

Donc: $ (1-3x)A(x) = \frac{1}{1-4x} \Leftrightarrow A(x) = \frac{1}{(1-4x)(1-3x)} \Leftrightarrow A(x) = \frac{a}{1-4x} + \frac{b}{1-3x} $

Si on résout l'équation $1 = a(1-3x) + b(1-4x)$ on trouve $b = -3$ et $a = 4$. 

Donc: 

\begin{align*}
A(x) &= \frac{4}{1-4x} - \frac{3}{1-3x} \\
	&= 4 \cdot \frac{1}{1-4x} - 3 \cdot \frac{1}{1-3x}\\
	&= 4 \cdot ( \sum_{n=0}^{\infty} (4x)^n ) - 3 \cdot ( \sum_{n=0}^{\infty} (3x)^n ) \\
	&= \sum_{n=0}^{\infty} (4^{n+1} - 3^{n+1}) x^n
\end{align*}

Par conséquent: $a_n = 4^{n+1} - 3^{n+1} \qquad \forall n \geq 0$

\newpage
%--------------------

\begin{exo}
Un collectionneur excentrique rafolle des pavages de rectangles $2 \times n$ par des dominos verticaux $2 \times 1$ et horizontaux $1 \times 2$. Il paye sans h\'esiter $4$\euro{} par domino vertical et $1$\euro{} par domino horizontal. Pour combien de pavages sera-t-il pr\^et \`a payer $n$\euro~?
\end{exo}


<La résolution de cet exo prends genre 3 ou 4 pages de calculs et explications donc je vais clairement pas la retranscrire>
%--------------------