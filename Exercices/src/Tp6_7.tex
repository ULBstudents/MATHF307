\section{Séance 6 et 7}

\begin{exo}
Soient $A$ et $B$ deux ensembles finis avec $|A|=a$ et $|B|=b$ ($a,b \in \mathbb{N}$). Que valent:
\begin{enumerate}[(i)]
\item $|A\times B|$,
\item $|B^A|$ o\`u $B^A:=\{f:A \rightarrow B\}$,
\item $|\{f:A \rightarrow B: f \mathrm{~est~une~injection~de~} A \mathrm{~dans~}B\}|$,
\item $|\mathrm{Sym}\,A|$ o\`u $\mathrm{Sym}\,A$ est l'ensemble des permutations de $A$.
\end{enumerate}
\end{exo}

\begin{enumerate}[(i)]
	\item $a\cdot b$
	\item $b^a$
	\item voir cahier
	\item $a!$
\end{enumerate}

%----------------------------------------------

\begin{exo}
Quels sont les ensembles $F$ non vides ayant la propri\'et\'e suivante:
\begin{enumerate}[(i)]
\item pour tout ensemble $X$, $|F^X|=1$?
\item pour tout ensemble $Y$, $|Y^F|=1$?
\end{enumerate}
\end{exo}

\begin{enumerate}[(i)]
	\item $|F| = 1 $
	\item Ensemble vide (mais pas possible par énoncé)
\end{enumerate}

%----------------------------------------------

\begin{exo}
Soient $f:A \rightarrow B$ et $g:B \rightarrow C$ deux fonctions. D\'emontrer:
\begin{enumerate}[(i)]
\item $g \circ f$ injective $\Rightarrow$ $f$ injective;
\item $g \circ f$ surjective $\Rightarrow$ $g$ surjective;
\item $g \circ f$ bijective $\Rightarrow$ ($f$ injective et $g$ surjective).
\end{enumerate}
\end{exo}

\begin{enumerate}[(i)]
	\item Si f non injective, deux éléments $a_{1}$ et $a_{2}$ différents de A vont être envoyés par f sur un élément b de B. De plus, ces deux éléments vont être envoyés par g o f sur un même élément c de C, car g ( f ($a_1$)) = g( b ) = c = g( b) = g ( f ( $a_2$))
	\item On sait que $\forall c \in C , \exists a \in A$ tel que g o f (a) = c. 
		On veut montrer que g est surjective. C'est à dire que $\forall c \in C, \exists b \in B$ tel que g (b) = c. 
		Ceci est vérifié en prenant b = f(a).
	\item Implication de (i) et (ii)
\end{enumerate}

%----------------------------------------------

\begin{exo}
Donner une preuve bijective de l'identit\'e de somme parall\`ele ${k \choose k} + {k+1 \choose k} + \cdots + {m \choose k} = {m+1 \choose k+1}$.
\end{exo}

%----------------------------------------------

\begin{exo}
Donner deux d\'emonstrations de
$$
\sum_{k=0}^n {n \choose k} = 2^n\ .
$$
\end{exo}

%----------------------------------------------

\begin{exo}
Qu'obtient-on comme identit\'e sur les coefficients binomiaux en \'ecrivant
$$
(x+y)^{2n} = (x+y)^n(x+y)^n\ ?
$$
\end{exo}

%----------------------------------------------

\begin{exo}
Qu'obtient-on en d\'erivant la formule du bin\^ome~?
\end{exo}

%----------------------------------------------

\begin{exo}
Trouver le nombre de solutions de l'\'equation $x + y + z + w = 15$, dans les naturels.
\end{exo}

%----------------------------------------------

\begin{exo} 
Combien l'\'equation
$$
x + y + z + t + u = 60
$$
poss\`ede-t-elle de solutions enti\`eres $(x,y,z,t,u)$ telles que
$$
x > 0\ ,\quad y \geqslant 9\ , \quad z > -2\ , \quad t \geqslant 0 \quad \textrm{ et }  \quad u > 10 \quad ?
$$
\end{exo}

%----------------------------------------------

\begin{exo} 
Trouver le nombre de solutions de l'in\'equation
$$
x + y + z + t \leqslant 6
$$
%
\begin{enumerate}[(i)]
\item dans les naturels;
\item dans les entiers $>0$;
\item dans les entiers, avec comme contraintes suppl\'ementaires $x > 2$, $y > -2$, $z > 0$ et $t > -3$.
\end{enumerate}
\end{exo}

%----------------------------------------------

\begin{exo} 
Avec les lettres du mot MISSISSIPPI, combien peut-on \'ecrire de mots diff\'erents de 11 lettres~?
\end{exo}

1 M 4 I 4 S 2 P
	Mots de 11 lettres
		(11!)/ ( (4!)*(4!)*(2!)*(1!)


%----------------------------------------------

\begin{exo} 
Avec les lettres du mot 
%
\begin{center}
H\,U\,M\,U\,H\,U\,M\,U\,N\,U\,K\,U\,N\,U\,K\,U\,A\,P\,U\,A\,A
\end{center}
(``poisson'' en hawa\"\i{}en), combien peut-on \'ecrire de mots diff\'erents de 21 lettres ne comprenant pas deux lettres U c\^ote \`a c\^ote~?
\end{exo}

Faire mots de 12 lettres sans U. Rajouter proba de foutre les U dans les 13 places qui restent pour faire des mots de 21 lettres. Donc, (13 9) (13 chouk 9).

%----------------------------------------------

\begin{exo} 
Si $0 \leqslant m \leqslant n$, que vaut
$$
\sum_{k=m}^n {k \choose m}{n \choose k}\quad ?
$$
(Hint~: essayer une preuve bijective.)
\end{exo}

%----------------------------------------------

\begin{exo} 
Si on jette simultan\'ement $n$ d\`es identiques, combien de r\'esultats diff\'erents peut-on obtenir~? (Deux r\'esultats sont consid\'er\'es comme \'equivalents s'ils ont le m\^eme nombre de 1, le m\^eme nombre de 2, \ldots, le m\^eme nombre de 6.)
\end{exo}