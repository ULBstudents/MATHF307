\section{Séance 8 et 9}

\begin{exo}
De combien de fa\c{c}ons diff\'erentes peut-on monter un escalier de 30 marches, si on monte \`a chaque pas soit d'une seule marche soit de deux marches \`a la fois~?
\end{exo}

\begin{exo} 
Que vaut le d\'eterminant de la matrice $n \times n$ 
$$
\left( 
\begin{array}{rrrrrrrr}
1 &-1 &0  &0  &\cdots &0 &0 &0\\
1 & 1 &-1 &0  &\cdots &0 &0 &0\\
0 & 1 &1  &-1 &\cdots &0 &0 &0\\
\vdots & \vdots & \vdots & \vdots && \vdots & \vdots &\vdots\\
0 & 0 & 0 & 0 & \cdots & ~1 &~1 &-1\\
0 & 0 & 0 & 0 & \cdots & 0 &1 &1\\
\end{array}
\right) \qquad ?
$$
\end{exo}

\begin{exo} 
Que vaut
$$
\lim_{n \to \infty} \frac{F_{n+1}}{F_n} \quad ?
$$
\end{exo}

\begin{exo} 
Prouver que, pour tout entier $n \geqslant 1$, 
$$
\varphi^n = F_n \cdot \varphi + F_{n-1}\ ,
$$
o\`u $\varphi := \frac{1+\sqrt{5}}{2}$ est le {\DEF nombre d'or}.
\end{exo}

\begin{exo} 
Prouver que, pour tout entier $n \geqslant 3$,
$$
F_n > \varphi^{n-2}
$$
\end{exo}

\begin{exo}
R\'esoudre les r\'ecurrences 
%
\begin{enumerate}[(i)]
\item $a_n = \frac{1}{2} a_{n-1} + 1$ pour $n \geqslant 1$,\hfill
 $a_0 = 1$

\item $a_n = 5a_{n-1} - 6a_{n-2}$ pour $n \geqslant 2$,\hfill
 $a_0 = -1$, \quad $a_1 = 1$

\item $a_n = 6a_{n-1} - 9a_{n-2}$ pour $n \geqslant 2$,\hfill
 $a_0 = 1$, \quad $a_1 = 9$

\item $a_n = 4a_{n-1} - 3a_{n-2} + 2^n$ pour $n \geqslant 2$,\hfill
$a_0 = 1$, \quad $a_1 = 11$
\end{enumerate}
\end{exo}

\begin{exo}
R\'esoudre les r\'ecurrences 
%
\begin{enumerate}[(i)]
\item $a_{n+2} = 3 a_{n+1} + 4 a_{n}$ pour $n \geqslant 0$,\hfill
      $a_0 = 1$, \quad $a_1 = 3$

\item $a_{n+3} - 6 a_{n+2} + 11 a_{n+1} - 6 a_{n}=0$ pour $n \geqslant 0$,\hfill
      $a_0 = 2$, \quad $a_1 = 0$, \quad $a_2 = -2$

\item $a_{n+3} = 3 a_{n+1} - 2 a_{n}$ pour $n \geqslant 0$,\hfill
      $a_0 = 1$, \quad $a_1 = 0$, \quad $a_2 = 0$
      
\item $a_{n+3} + 3 a_{n+2} + 3 a_{n+1} + a_n = 0$

\item $a_{n+4} + 4 a_{n} = 0$     
\end{enumerate}
\end{exo}

\begin{exo}
R\'esoudre la r\'ecurrence
$$
\begin{array}{ll}
a_{n+2} - (2 \cos \alpha) a_{n+1} + a_n = 0\quad \forall n \geqslant 0\\
a_1 = \cos \alpha, \quad a_2 = \cos 2 \alpha
\end{array}
$$
\end{exo}

\begin{exo}
R\'esoudre les r\'ecurrences
%
\begin{enumerate}[(i)]
\item $a_n + 2 a_{n-1} = n+3$ pour $n \geqslant 1$\hfill
      $a_0 = 3$
      
\item $a_{n+2} + 8 a_{n+1} - 9 a_{n} = 8 \cdot 3^{n+1}$ pour $n \geqslant 0$\hfill
      $a_0 = 2$, \quad $a_1 = -6$
      
\item $a_{n+2} - 6 a_{n+1} + 9 a_{n} = 2^n + n$ pour $n \geqslant 0$

\item $n a_n = (n+3) a_{n-1} + n^2 + n$ pour $n \geqslant 1$            
\end{enumerate}
\end{exo}

\begin{exo}
Que vaut le d\'eterminant de la matrice $n \times n$
$$
\left( 
\begin{array}{rrrrrrrr}
2 &1 &0  &0  &\cdots &0 &0 &0\\
1 & 2 &1 &0  &\cdots &0 &0 &0\\
0 & 1 &2  &1 &\cdots &0 &0 &0\\
\vdots & \vdots & \vdots & \vdots && \vdots & \vdots &\vdots\\
0 & 0 & 0 & 0 & \cdots & 1 &2 &1\\
0 & 0 & 0 & 0 & \cdots & 0 &1 &2\\
\end{array}
\right) \qquad ?
$$
\end{exo}

\begin{exo}
Avec l'alphabet $\{A,B,C\}$, combien peut-on \'ecrire de mots de $n$ lettres dans lesquels on ne trouve pas
%
\begin{enumerate}[(i)]
\item deux lettres $A$ c\^ote-\`a-c\^ote~?
\item deux lettres $A$ ni deux lettres $B$ c\^ote-\`a-c\^ote~?
\item deux lettres $A$ ni deux lettres $B$ ni deux lettres $C$ c\^ote-\`a-c\^ote~?
\end{enumerate}
\end{exo}

\begin{exo}
Donner le comportement asymptotique des suites $T(n)$ pour chacune des r\'ecurrences suivantes~:
%
\begin{enumerate}[(i)]
\item $T(n) = 2T(\lceil n/2 \rceil) + n^2$
\item $T(n) = T(\lfloor 9n/10 \rfloor) + n$
\item $T(n) = 16T(\lceil n/4 \rceil) + n^2$
\item $T(n) = 7 T(\lceil n/3 \rceil) + n^2$
\item $T(n) = 7 T(\lceil n/2 \rceil) + n^2$
\item $T(n) = 2 T(\lfloor n/4 \rfloor) + \sqrt{n}$
\item $T(n) = T(n-1) + n$
\item $T(n) = T(\lfloor \sqrt{n} \rfloor) + 1$
\end{enumerate}
\end{exo}

\begin{exo}
R\'esoudre la r\'ecurrence
$$
\begin{array}{l}
a_n = \sqrt{a_{n-1} a_{n-2}} \quad \forall n \geqslant 2\\
a_0 = 1, \quad a_1 = 2
\end{array}
$$
\end{exo}

\begin{exo} (Examen ao\^ut 2011.)
Combien y a-t-il de matrices $2 \times n$ \`a coefficients entiers v\'erifiant les deux conditions suivantes~?
%
\begin{itemize}
\item Dans chacune des deux lignes, chacun des entiers $1$, $2$, \ldots, $n$ appara\^\i{}t une et une seule fois.
\item Dans chacune des $n$ colonnes, les deux coefficients diff\`erent d'au plus $1$.
\end{itemize}
\end{exo}

\begin{exo} (Examen ao\^ut 2011.)
Soient $x$ et $y$ deux naturels de $2n$ bits, c'est-\`a-dire dont l'\'ecriture binaire occupe au plus $2n$ bits. %Supposons que $x = \sum_{i=0}^{2n-1} x_i 2^i$ et $y = \sum_{i=0}^{2n-1} y_i 2^i$ sont les \'ecritures binaires de $x$ et $y$. Cela peut s'\'ecrire aussi $x = (x_{2n-1}x_{2n-2}\cdots{}x_1x_0)_2$ et $y =  (y_{2n-1}y_{2n-2}\cdots{}y_1y_0)_2$. Posons maintenant $X_0 := (x_{n-1}x_{n-2}\cdots{}x_1x_0)_2$, $X_1 := (x_{2n-1}x_{2n-2}\cdots{}x_{n+1}x_n)_2$, $Y_0 := (y_{n-1}y_{n-2}\cdots{}y_1y_0)_2$ et $Y_1 := (y_{2n-1}y_{2n-2}\cdots{}y_{n+1}y_n)_2$. Donc $X_0$, $X_1$, $Y_0$ et $Y_1$ sont quatre naturels de $n$ bits tels que $x = 2^n X_1 + X_0$ et $y = 2^n Y_1 + Y_0$.\medskip
Soient $X_0$, $X_1$, $Y_0$ et $Y_1$ quatre naturels de $n$ bits tels que $x = 2^n X_1 + X_0$ et $y = 2^n Y_1 + Y_0$.\medskip

\begin{enumerate}[a)]
\item V\'erifier que
%
$$
xy = (2^{2n} + 2^n) X_1 Y_1 + 2^n (X_1 - X_0)(Y_0 - Y_1) + (2^n + 1)X_0Y_0\ .
$$

\item Consid\'erons l'algorithme r\'ecursif qui multiplie les naturels $x$ et $y$ en appliquant l'\'equation ci-dessus. Soit $f(n)$ le nombre d'op\'erations \underline{simples} (additions ou soustractions de bits, d\'ecalages, comparaisons\ldots{} etc) n\'ecessaires pour le calcul r\'ecursif du produit $xy$ par le biais de cette \'equation. Ecrire une relation de r\'ecurrence du type ``diviser pour r\'egner'' pour $f(n)$. (Il n'est pas n\'ecessaire de calculer avec pr\'ecision le nombre d'op\'erations simples requises, calculer ce nombre \`a une constante pr\`es est suffisant.)

\item Sur base de cette r\'ecurrence, d\'eterminer le comportement asymptotique de $f(n)$.
\end{enumerate}
\end{exo}

\begin{exo} (Difficile.)
R\'esoudre la r\'ecurrence (discuter en fonction de $a_0$)
$$
a_n = a_{n-1}^2 + 2 \quad \forall n \geqslant 1
$$
(Hint~: poser $a_n = b_n + 1/b_n$.)
\end{exo}

\begin{exo} (Difficile.)
Montrer que la solution de la r\'ecurrence 
$$
\begin{array}{l}
a_n = \sin (a_{n-1}) \quad \forall n \geqslant 1\\
a_0 = 1
\end{array}
$$
v\'erifie $\lim_{n \to \infty} a_n = 0$ et $a_n = O(1/\sqrt{n})$.

(Hint~: poser $b_n = 1/a_{n}$.)
\end{exo}