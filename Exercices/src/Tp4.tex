\documentclass[11pt, a4paper]{article} 

%\parindent 5mm 
%\parskip 0mm 
%\addtolength{\textwidth}{3 truecm} 
%\addtolength{\textheight}{1 truecm} 
%\setlength{\voffset}{-.6 truecm} 
%\setlength{\hoffset}{-1.3 truecm} 

\usepackage[lmargin=25mm,rmargin=25mm,tmargin=20mm,bmargin=20mm]{geometry}

\usepackage[frenchb]{babel} 
\usepackage{amsthm, amssymb, amsmath} 
\usepackage[T1]{fontenc} 
\usepackage{tikz, pgf} 
\usepackage{graphicx} 
\usepackage{xcolor} 
\usepackage{url, enumerate} 
\usepackage{epsfig}


\begin{document} 

\thispagestyle{empty}

\title{MATH-F-307: S\'eance d'exercices 4} 
\author{} 
\date{16 octobre 2015}

\theoremstyle{plain} 
\newtheorem*{theo}{Th\'eor\`eme}
%\newtheorem{defi}[theo]{D\'efinition} 
%\newtheorem{lemme}[theo]{Lemme} 
%\newtheorem{corol}[theo]{Corollaire} 
%\newtheorem{prop}[theo]{Proposition} 
%\newtheorem{fait}[theo]{Fait} 
%\newtheorem{aff}{Affirmation} 
%\newtheorem*{rapp}{Rappel} 
%\newtheorem{conj}[theo]{Conjecture} 

\theoremstyle{definition} 
%\newtheorem{probl}[theo]{Probl\`eme ouvert} 
%\newtheorem*{ex}{Exemple} 
%\newtheorem*{rem}{Remarque} 
\newtheorem{exo}{Exercice}

\newcommand{\R}{\mathbb{R}} 
\newcommand{\Z}{\mathbb{Z}} 
\newcommand{\N}{\mathbb{N}} 

%\renewcommand{\theenumi}{(\alph{enumi})}
%\renewcommand{\labelenumi}{\theenumi}
\renewcommand{\FrenchLabelItem}{\textbullet}

\maketitle


\begin{exo}
L'ensemble $\{2^m | m \in \Z\}$ forme-t-il un groupe lorsqu'il est muni de la multiplication usuelle?
\end{exo}

\vspace*{0.8cm}
\begin{exo}
L'ensemble $(\Z/2\Z)^n=\{(x_1,x_2,\ldots,x_n)|x_1,x_2,\ldots,x_n\in \Z/2\Z\}$ avec l'addition d\'efinie par $$(x_1,x_2,\ldots,x_n)+(y_1,y_2,\ldots,y_n)=(x_1+y_1,x_2+y_2,\ldots,x_n+y_n)$$
(o\`u $x_i+y_i$ est le r\'esultat d'une addition modulo $2$) forme-t-il un groupe?
\end{exo}

\vspace*{0.8cm}
\begin{exo}
En appliquant l'algorithme d'Euclide \`a $a$ et $b$ ci-dessous, calculer :
\begin{itemize}
\item[-] le PGCD($a,b$),
\item[-] $x$ et $y$ tels que $ ax + by = \text{PGCD}(a,b),$
\end{itemize}
Les diff\'erentes valeurs de $a$ et $b$ sont :
\begin{enumerate}[(i)]
\item $a = 12, b = 34$,
\item $a = 13, b = 34$,
\item $a = 13, b = 31$,
\end{enumerate}
\end{exo}

\vspace*{0.8cm}
\begin{exo} 
\begin{enumerate}[(i)]
\item Trouver un entier $x$ tel que le reste de la division de $50x$ par $71$ donne $1$.
\item Trouver un entier $x$ tel que le reste de la division de $50x$ par $71$ donne $63$.
\item Trouver un entier $x$ tel que le reste de la division de $43x$ par $64$ donne $1$.
\end{enumerate}
\end{exo}

\vspace*{0.8cm}
\begin{exo}
Dans le syst\`eme RSA, prenons $p=11$, $q=13$ et $e=7$. Que vaut alors $s$? Si $99$ est le message \`a coder, quel est le message crypt\'e? V\'erifier en d\'ecriptant le message.
\end{exo}

\thispagestyle{empty}

\end{document}