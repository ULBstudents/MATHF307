
\section{Séance 4}

\begin{exo}
L'ensemble $\{2^m | m \in \Z\}$ forme-t-il un groupe lorsqu'il est muni de la multiplication usuelle?
\end{exo}

Soit $X=2^{m}, m \in \Z$ et $(X,.)$ le groupe à analyser. \\

$\forall x,y \in \Z : 2^{x} * 2^{y} = 2^{x+y} \in X$ car $(x+y) \in \Z$. L'ensemble X forme donc bien un groupe lorsqu'il est muni de la multiplication. 

%------------------------------------------------

\vspace*{0.8cm}
\begin{exo}
L'ensemble $(\Z/2\Z)^n=\{(x_1,x_2,\ldots,x_n)|x_1,x_2,\ldots,x_n\in \Z/2\Z\}$ avec l'addition d\'efinie par $$(x_1,x_2,\ldots,x_n)+(y_1,y_2,\ldots,y_n)=(x_1+y_1,x_2+y_2,\ldots,x_n+y_n)$$
(o\`u $x_i+y_i$ est le r\'esultat d'une addition modulo $2$) forme-t-il un groupe?
\end{exo}

Il faut tester si les 3 propriétés d'un groupe sont respectées.

\begin{enumerate}
\item Associativité: Chaque composante est calculée avec la forme $x_{i}+y_{i}, \forall i \in \{ 1,2,...,n \}$. $\Z_{2}$ est associatif, l'adition est faite composante par composante, donc $Z_{2}^{n}$ est associatif. Il faut donc à présent montrer que $(x_{i}+y_{i}) + z_{i} = x_{i} + (y_{i} + z_{i})$.
\end{enumerate}

%------------------------------------------------

\vspace*{0.8cm}
\begin{exo}
En appliquant l'algorithme d'Euclide \`a $a$ et $b$ ci-dessous, calculer :
\begin{itemize}
\item[-] le PGCD($a,b$),
\item[-] $x$ et $y$ tels que $ ax + by = \text{PGCD}(a,b),$
\end{itemize}
Les diff\'erentes valeurs de $a$ et $b$ sont :
\begin{enumerate}[(i)]
\item $a = 12, b = 34$,
\item $a = 13, b = 34$,
\item $a = 13, b = 31$,
\end{enumerate}
\end{exo}

%-----------------------------------------------------------------

\vspace*{0.8cm}
\begin{exo} 
\begin{enumerate}[(i)]
\item Trouver un entier $x$ tel que le reste de la division de $50x$ par $71$ donne $1$.
\item Trouver un entier $x$ tel que le reste de la division de $50x$ par $71$ donne $63$.
\item Trouver un entier $x$ tel que le reste de la division de $43x$ par $64$ donne $1$.
\end{enumerate}
\end{exo}

%-----------------------------------------------------------------

\vspace*{0.8cm}
\begin{exo}
Dans le syst\`eme RSA, prenons $p=11$, $q=13$ et $e=7$. Que vaut alors $s$? Si $99$ est le message \`a coder, quel est le message crypt\'e? V\'erifier en d\'ecriptant le message.
\end{exo}

\thispagestyle{empty}