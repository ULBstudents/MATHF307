
\section{Séance 4}

\begin{exo}
L'ensemble $\{2^m | m \in \Z\}$ forme-t-il un groupe lorsqu'il est muni de la multiplication usuelle?
\end{exo}

Soit $X=2^{m}, m \in \Z$ et $(X,.)$ le groupe à analyser.

\begin{enumerate}
\item $\forall x,y \in \Z : 2^{x} \cdot 2^{y} = 2^{x+y} \in X$ car $(x+y) \in \Z$
\item (Associativité) $\forall m,n,0 \in \Z: 2^m \cdot (2^n \cdot 2^0) = 2^m \cdot 2^{n+0} = 2^{m+n+0} = 2^{m+n} \cdot 2^0 = (2^m \cdot 2^n) \cdot 2^0$
\item (Élément neutre) $2^m \cdot 2^0 = 2^{m+0} = 2^m = 2^0 \cdot 2^m = 1 \cdot 2^m$
\item (Opposé) $2^m \cdot 2^{-m} = 2^{m+(-m)} = 2^0 = 1$  
\end{enumerate}

L'ensemble forme donc bien un groupe.

%------------------------------------------------

\vspace*{0.8cm}
\begin{exo}
L'ensemble $(\Z/2\Z)^n=\{(x_1,x_2,\ldots,x_n)|x_1,x_2,\ldots,x_n\in \Z/2\Z\}$ avec l'addition d\'efinie par $$(x_1,x_2,\ldots,x_n)+(y_1,y_2,\ldots,y_n)=(x_1+y_1,x_2+y_2,\ldots,x_n+y_n)$$
(o\`u $x_i+y_i$ est le r\'esultat d'une addition modulo $2$) forme-t-il un groupe?
\end{exo}

Il faut tester si les 3 propriétés d'un groupe sont respectées.

\begin{enumerate}
\item (Associativité) OK: l'addition dans $\Z_2$ est associative donc elle l'est également pour chaque composante.
\item (Élément neutre) $(0,0,\ldots,0) \in \Z^{n}_{2}$ 
\item (Opposé) $\forall (x_1, x_2, \ldots, x_n) \in \Z^{n}_{2}: (x_1, x_2, \ldots, x_n) + (x_1, x_2, \ldots, x_n) = (2x_1, 2x_2, \ldots, 2x_n) = (0,0,\ldots,0)$. Donc chaque élément est son propre inverse.
\end{enumerate}

%------------------------------------------------

\vspace*{0.8cm}
\begin{exo}
En appliquant l'algorithme d'Euclide \`a $a$ et $b$ ci-dessous, calculer :
\begin{itemize}
\item[-] le PGCD($a,b$),
\item[-] $x$ et $y$ tels que $ ax + by = \text{PGCD}(a,b),$
\end{itemize}
Les diff\'erentes valeurs de $a$ et $b$ sont :
\begin{enumerate}[(i)]
\item $a = 12, b = 34$,
\item $a = 13, b = 34$,
\item $a = 13, b = 31$,
\end{enumerate}
\end{exo}

<MISSING>

\newpage
%-----------------------------------------------------------------

\vspace*{0.8cm}
\begin{exo} 
\begin{enumerate}[(i)]
\item Trouver un entier $x$ tel que le reste de la division de $50x$ par $71$ donne $1$.
\item Trouver un entier $x$ tel que le reste de la division de $50x$ par $71$ donne $63$.
\item Trouver un entier $x$ tel que le reste de la division de $43x$ par $64$ donne $1$.
\end{enumerate}
\end{exo}

<MISSING>

%-----------------------------------------------------------------

\vspace*{0.8cm}
\begin{exo}
Dans le syst\`eme RSA, prenons $p=11$, $q=13$ et $e=7$. Que vaut alors $s$? Si $99$ est le message \`a coder, quel est le message crypt\'e? V\'erifier en d\'ecriptant le message.
\end{exo}

<MISSING>

\thispagestyle{empty}