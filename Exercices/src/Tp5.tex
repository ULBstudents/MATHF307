
\section{Séance 5}

\begin{exo}
Montrer le r\'esultat suivant: si $a\equiv b ~(mod~n)$ et $c\equiv d ~(mod~n)$, alors
$$ a+c \equiv b+d ~(mod~n) ~\mathrm{et}~ a.c \equiv b.d ~(mod~n).$$
\end{exo}

\vspace*{0.8cm}
\begin{exo}
Montrer que, si $a\equiv b ~(mod~n)$, alors $$a+c\equiv b+c ~(mod~n) ~\forall c \in \Z$$ et $$a.c\equiv b.c ~(mod~n) ~\forall c \in \Z.$$ 
\end{exo}

\vspace*{0.8cm}
\begin{exo}
Prouver que, si $a\equiv b ~(mod~n)$, alors $a^k\equiv b^k ~(mod~n)$ pour tout entier $k>0$.
\end{exo}

\vspace*{0.8cm}
\begin{exo}
Trouver toutes les solutions aux congruences suivantes:
\begin{itemize}
\item $2x \equiv 3 ~(mod~4)$ avec $x \in \Z/4\Z$;
\item $2x \equiv 2 ~(mod~4)$ avec $x \in \Z/4\Z$;
\item $2x \equiv 3 ~(mod~5)$ avec $x \in \Z/5\Z$.
\end{itemize}
Que pouvez-vous en d\'eduire?
\end{exo}

\vspace*{0.8cm}
\begin{exo}
Soient $a,b$ deux entiers. Montrer que $$a\Z \cap b\Z = ppcm(a,b)\Z$$ et $$ a\Z + b\Z = pgcd(a,b)\Z.$$
\end{exo}

\vspace*{0.8cm}
\begin{exo}
Montrer que $$\Z/3\Z \times \Z/3\Z \ncong \Z/9\Z$$ mais que $$\Z/2\Z \times \Z/3\Z \cong \Z/6\Z.$$

\end{exo}