
\section{Séance 5}

\begin{exo}
Montrer le r\'esultat suivant: si $a\equiv b ~(mod~n)$ et $c\equiv d ~(mod~n)$, alors
$$ a+c \equiv b+d ~(mod~n) ~\mathrm{et}~ a.c \equiv b.d ~(mod~n).$$
\end{exo}

\begin{enumerate}[$\cdot$]
\item $n|a-b$ et $n|c-d \Rightarrow n|(a-b+c-d) \Leftrightarrow n|a+c-(b+d) \Leftrightarrow a+c \equiv b+d(mod\;n)$ 
\item $ac-bd = ac-bc+bc-bd = c(a-b) + b(c-d) \Rightarrow n|(c(a-b)+b(c-d)) \Rightarrow n|ac-bd \Rightarrow ac \equiv bd(mod\;n)$  
\end{enumerate}

%-----------------------------------------------------------------

\vspace*{0.8cm}
\begin{exo}
Montrer que, si $a\equiv b ~(mod~n)$, alors $$a+c\equiv b+c ~(mod~n) ~\forall c \in \Z$$ et $$a.c\equiv b.c ~(mod~n) ~\forall c \in \Z.$$ 
\end{exo}

$\forall c \in \Z, c \equiv c(mod\;n)$

Utiliser l'exércice précédent pour prouver.

%-----------------------------------------------------------------

\vspace*{0.8cm}
\begin{exo}
Prouver que, si $a\equiv b ~(mod~n)$, alors $a^k\equiv b^k ~(mod~n)$ pour tout entier $k>0$.
\end{exo}

$a^{k}\equiv b^{k} ~(mod~n) \Rightarrow b^{k} - a^{k} = xn \Leftrightarrow ln(b^{k} - a^{k}) = ln(xn) \Leftrightarrow \frac{ln(b^{k})}{ln(a^k)} = ln(xn) \Leftrightarrow \frac{\cancel{k}ln(b)}{\cancel{k}ln(a)} = ln(xn) \Leftrightarrow ln(b) - ln(a) = ln(xn) \Leftrightarrow b - a = xn \Rightarrow a\equiv b ~(mod~n)$

%-----------------------------------------------------------------

\vspace*{0.8cm}
\begin{exo}
Trouver toutes les solutions aux congruences suivantes:
\begin{itemize}
\item $2x \equiv 3 ~(mod~4)$ avec $x \in \Z/4\Z$;
\item $2x \equiv 2 ~(mod~4)$ avec $x \in \Z/4\Z$;
\item $2x \equiv 3 ~(mod~5)$ avec $x \in \Z/5\Z$.
\end{itemize}
Que pouvez-vous en d\'eduire?
\end{exo}

%-----------------------------------------------------------------

\vspace*{0.8cm}
\begin{exo}
Soient $a,b$ deux entiers. Montrer que $$a\Z \cap b\Z = ppcm(a,b)\Z$$ et $$ a\Z + b\Z = pgcd(a,b)\Z.$$
\end{exo}

\textbf{Sous-question 1:} 

Soit $m = ppcm(a,b)$. Nous pouvons en déduire 3 propriétés:

\begin{enumerate}
	\item $\frac{a}{m}$
	\item $\frac{b}{m}$
	\item si $\frac{a}{z}$ et $\frac{b}{z}$, alors $\frac{m}{z}$ 
\end{enumerate}

Nous voulons donc prouver que $a\Z \cap b\Z = m\Z$. Pour ce faire, nous devons montrer que l'un est inclus dans l'autre, et vice-versa. 

$(\subseteq)$ Soit $z \in a\Z \cap b\Z $, i.e. $\exists k,{k}' \in \Z $ \hspace{1cm} $z = ak = b{k}' \Leftrightarrow \frac{a}{z}$ et $\frac{b}{z}$ 

Montrer que $z \in m\Z$. 

Par la propriété 3, $\frac{m}{z} \Leftrightarrow z \in m\Z$ 

$(\supseteq)$ Soit $z \in m\Z $, i.e. $\frac{m}{z}$ 

Montrer que $z \in a\Z \cap b\Z$, i.e. $\frac{a}{z}$ et $\frac{b}{z}$. 

\begin{minipage}{.5\textwidth}
	Par la propriété 1, $\frac{a}{m}$

	Par la propriété 2, $\frac{b}{m}$
\end{minipage}
\begin{minipage}{.5\textwidth}
	$\frac{m}{z} \Rightarrow \frac{a}{z}$ et $\frac{b}{z}$ 
\end{minipage}

\vspace*{0.3cm}

\textbf{Sous-question 2:} 

\vspace*{0.8cm}

%-----------------------------------------------------------------

\begin{exo}
Montrer que $$\Z/3\Z \times \Z/3\Z \ncong \Z/9\Z$$ mais que $$\Z/2\Z \times \Z/3\Z \cong \Z/6\Z.$$

\end{exo}