\section{Séance 8 et 9}

\begin{exo}
De combien de fa\c{c}ons diff\'erentes peut-on monter un escalier de 30 marches, si on monte \`a chaque pas soit d'une seule marche soit de deux marches \`a la fois~?
\end{exo}

%-----------------------------------------------------------------

\begin{exo} 
Que vaut le d\'eterminant de la matrice $n \times n$ 
\[
\left( 
\begin{array}{rrrrrrrr}
1 &-1 &0  &0  &\cdots &0 &0 &0\\
1 & 1 &-1 &0  &\cdots &0 &0 &0\\
0 & 1 &1  &-1 &\cdots &0 &0 &0\\
\vdots & \vdots & \vdots & \vdots && \vdots & \vdots &\vdots\\
0 & 0 & 0 & 0 & \cdots & ~1 &~1 &-1\\
0 & 0 & 0 & 0 & \cdots & 0 &1 &1\\
\end{array}
\right) \qquad ?
\]
\end{exo}

Soit $M_n$ la matrice de dimension $n\times n$ de l'exercice.

Si on calcule le déterminant selon la première ligne, on a 

\[|M_n| = \underbrace{1\cdot (-1)^{1+1} \cdot |M_{n-1}|}_{\text{En enlevant première ligne, première colonne}} + \underbrace{(-1) \cdot (-1)^{1+2} \cdot |X|}_{\text{En enlevant première ligne, deuxième colonne}} = |M_{n-1}| + |X|\].

(X étant la sous-matrice de $M_n$ à laquelle on a enlevé la première ligne et première colonne (DESSIN). Nous pouvons voir que pour calculer le déterminant de X, en appliquant la même méthode on a $1 \cdot (-1)^{1+1} \cdot |M_{n-2}| = |M_{n-2}|$.

Maintenant qu'on a le déterminant de X, nous pouvons le remplacer dans la première formule:

$|M_{n}| = |M_{n-1}| + |M_{n-2}|$

On retombe sur Fibo.

%-----------------------------------------------------------------

\begin{exo} 
Que vaut
\[
\lim_{n \to \infty} \frac{F_{n+1}}{F_n} \quad ?
\]
\end{exo}

%-----------------------------------------------------------------

\begin{exo} 
Prouver que, pour tout entier $n \geqslant 1$, 
\[
\varphi^n = F_n \cdot \varphi + F_{n-1}\ ,
\]
o\`u $\varphi := \frac{1+\sqrt{5}}{2}$ est le {\DEF nombre d'or}.
\end{exo}

%-----------------------------------------------------------------

\begin{exo}
R\'esoudre les r\'ecurrences 
%
\begin{enumerate}[(i)]
\item $a_n = \frac{1}{2} a_{n-1} + 1$ pour $n \geqslant 1$,\hfill
 $a_0 = 1$

\item $a_n = 5a_{n-1} - 6a_{n-2}$ pour $n \geqslant 2$,\hfill
 $a_0 = -1$, \quad $a_1 = 1$

\item $a_n = 6a_{n-1} - 9a_{n-2}$ pour $n \geqslant 2$,\hfill
 $a_0 = 1$, \quad $a_1 = 9$

\end{enumerate}
\end{exo}

%-----------------------------------------------------------------

\begin{exo}
R\'esoudre les r\'ecurrences 
%
\begin{enumerate}[(i)]

\item $a_{n+3} = 3 a_{n+1} - 2 a_{n}$ pour $n \geqslant 0$,\hfill
      $a_0 = 1$, \quad $a_1 = 0$, \quad $a_2 = 0$
      
\item $a_{n+3} + 3 a_{n+2} + 3 a_{n+1} + a_n = 0$

\end{enumerate}
\end{exo}

%-----------------------------------------------------------------

\begin{exo}
R\'esoudre la r\'ecurrence
\[
\begin{array}{ll}
a_{n+2} - (2 \cos \alpha) a_{n+1} + a_n = 0\quad \forall n \geqslant 0\\
a_1 = \cos \alpha, \quad a_2 = \cos 2 \alpha
\end{array}
\]
\end{exo}

%-----------------------------------------------------------------

\begin{exo}
R\'esoudre les r\'ecurrences
%
\begin{enumerate}[(i)]
\item $a_n + 2 a_{n-1} = n+3$ pour $n \geqslant 1$\hfill
      $a_0 = 3$
      
\item $a_{n+2} + 8 a_{n+1} - 9 a_{n} = 8 \cdot 3^{n+1}$ pour $n \geqslant 0$\hfill
      $a_0 = 2$, \quad $a_1 = -6$
      
\item $a_{n+2} - 6 a_{n+1} + 9 a_{n} = 2^n + n$ pour $n \geqslant 0$

\end{enumerate}
\end{exo}

%-----------------------------------------------------------------

\begin{exo}
Avec l'alphabet $\{A,B,C\}$, combien peut-on \'ecrire de mots de $n$ lettres dans lesquels on ne trouve pas
%
\begin{enumerate}[(i)]
\item deux lettres $A$ c\^ote-\`a-c\^ote~?
\item deux lettres $A$ ni deux lettres $B$ c\^ote-\`a-c\^ote~?
\item deux lettres $A$ ni deux lettres $B$ ni deux lettres $C$ c\^ote-\`a-c\^ote~?
\end{enumerate}
\end{exo}

\begin{enumerate}[(i)]
\item

\item $a_n$ mots qui finissent par A, $b_n$ mots qui finissent par B, $c_n$ mots qui finissent par C.

\item 
\end{enumerate}

%-----------------------------------------------------------------

\begin{exo}
Donner le comportement asymptotique des suites $T(n)$ pour chacune des r\'ecurrences suivantes~:
%
\begin{enumerate}[(i)]
\item $T(n) = 2T(\lceil n/2 \rceil) + n^2$
\item $T(n) = 16T(\lceil n/4 \rceil) + n^2$
\item $T(n) = 7 T(\lceil n/2 \rceil) + n^2$
\item $T(n) = T(n-1) + n$
\end{enumerate}
\end{exo}

%-----------------------------------------------------------------

\begin{exo}
R\'esoudre la r\'ecurrence
\[
\begin{array}{l}
a_n = \sqrt{a_{n-1} a_{n-2}} \quad \forall n \geqslant 2\\
a_0 = 1, \quad a_1 = 2
\end{array}
\]
\end{exo}

%-----------------------------------------------------------------

\begin{exo} (Examen ao\^ut 2011.)
Combien y a-t-il de matrices $2 \times n$ \`a coefficients entiers v\'erifiant les deux conditions suivantes~?
%
\begin{itemize}
\item Dans chacune des deux lignes, chacun des entiers $1$, $2$, \ldots, $n$ appara\^\i{}t une et une seule fois.
\item Dans chacune des $n$ colonnes, les deux coefficients diff\`erent d'au plus $1$.
\end{itemize}
\end{exo}