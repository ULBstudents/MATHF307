
\section{Arithmétique Modulaire}

\subsection{Les entiers et la division euclidienne}

L'ensemble des entiers est noté $\Z$, il contient les entiers naturels ($\N$) et leur opposé. Il est naturellement muni de 2 opérations qui satisfont les propriétés suivantes:

\begin{enumerate}
\item \textbf{L'addition} +: $\Z \times \Z \rightarrow \Z : (a,b) \rightarrow a+b $\\
Propriétés: 
	\begin{enumerate}
		\item \textbf{Associativité} $(a+b)+c = a+(b+c)$, $\forall a,b,c \in \Z$
		\item \textbf{Élément neutre} $0 \in \Z$: $a+0=a=0+a$, $\forall a \in \Z$
		\item \textbf{Opposé} $\forall a \in \Z: \exists -a \in \Z$ tel que $a+(-a) =0=(-a)+a$
		\item \textbf{Commutativité} $\forall a,b \in \Z: a+b = b+a$	\\
	\end{enumerate}
On dit que $(\Z,+)$ est un groupe (a,b,c) commutatif (d).\\

\item \textbf{La multiplication} $\cdot$: $\Z \times \Z \rightarrow \Z : (a,b) \rightarrow a \cdot b$ \\
Propriétés:
	\begin{enumerate}
		\item \textbf{Associativité} $a\cdot(b\cdot c)=(a\cdot b)\cdot c$
		\item \textbf{Distributivité par rapport à l'addition} \\
			\begin{minipage}{.2\textwidth}
				\hspace{0.5cm}$a\cdot (b+c) = ab + ac$\\

				\hspace{0.5cm}$(a+b)\cdot c = ac + bc$\\
			\end{minipage}
			\begin{minipage}{.2\textwidth}
				\hspace{1cm}$\forall a,b,c \in \Z$\\
			\end{minipage}
		\item \textbf{Commutativité} $a \cdot b = b\cdot a, \forall a,b \in \Z$
		\item \textbf{Élément neutre} $1 \in \Z$: $1 \cdot a=a=a \cdot 1$, $\forall a \in \Z$
		\item $\forall a,b,c \in \Z: a\cdot c = a \cdot b \Rightarrow c=b$ \\
	\end{enumerate}
On dit que $(\Z,+,\cdot)$ est un anneau ($(\Z,+)$ est un groupe commutatif et $\cdot$ satisfait a et b) unital (d), commutatif (c) et intègre (e).\\
\end{enumerate}

On a aussi sur $\Z$ une relation d'ordre $\leq$ telle que:

\begin{enumerate}
	\item $\leq$ est un ordre total
	\item $\forall a,b,c \in \Z$, $a\leq b \Rightarrow a+c \leq b+c$
	\item $\forall a,b,c \in \Z$, $a\leq b$, $c\geq 0 \Rightarrow ac\leq bc$\\
\end{enumerate}

La valeur absolue est une application\\

\[ |\;|: \Z \rightarrow \N : a \rightarrow
  \begin{cases}
    a   & \quad \text{si } a\geq 0 \\
    -a  & \quad \text{si } a\leq 0 \\
  \end{cases}
\]

telle que: 

	\begin{enumerate}
		\item $\forall a \in \Z: |a| = 0$ ssi $a=0$
		\item $\forall a,b \in \Z: |a\cdot b| = |a| \cdot |b|$\\
	\end{enumerate}

Remarque: L'équation $ax = b, a,b\in \Z$ n'a pas toujours de solution dans $\Z$.\\

\begin{defn}
Soit $a,b \in \Z$, on dit que a divise b, et on note $a|b$, si $\exists c \in \Z$ tel que $ac=b$. On dit aussi que b est un multiple de a.
\end{defn}

\begin{prop}
| est une relation:

\begin{enumerate}
	\item \textbf{Réflexive} $\forall a \in \Z$: $a|a$
	\item \textbf{Transitive} $\forall a,b,c \in \Z$: $a|b$ et $b|c \Rightarrow a|c$
	\item \textbf{Anti-symétrique} $\forall a,b \in \Z$: $a|b$ et $b|a \Rightarrow a=\pm b$
\end{enumerate}
\end{prop}

\begin{thrm}[Division Euclidienne]
$\forall a,b \in \Z$, $b \neq 0$, $\exists$ des entiers uniques q et r tels que $a= bq + r$ et $0 \leq r < |b|$
\end{thrm}

<PAGES 3 À 6>

\subsection{Groupes, anneaux et entiers modulo n}

<PAGES 7 À 18>

\subsubsection{Relation de congruence}

\begin{defn}
Soient $a,b,k \in \Z, k \neq 0,1,-1$. On dit que a est congru à b modulo k et on note $a \equiv b(mod\;k)$ si $a-b \in k\Z$ (ou encore si $\overline{a} = \overline{b}$ dans $\Z/k\Z$). \\
\end{defn}

Propriétés:

\begin{enumerate}
\item La congruence modulo k est une relation d'équivalence.
	\begin{enumerate}[-]
	\item \textbf{Réflexivité} $\forall a \in \Z: a \equiv a(mod\;k)$
	\item \textbf{Symétrie} $\forall a,b \in \Z: a \equiv b(mod\;k) \Leftrightarrow b \equiv a(mod\;k)$
	\item \textbf{Transitivité} $\forall a,b,c \in \Z:$
				$ \left\{\begin{matrix}
				a \equiv b(mod\;k)\\ 
				b \equiv c(mod\;k)
				\end{matrix}\right.$ $\quad \Rightarrow a\equiv c(mod\;k)$
	\end{enumerate}

\item $\forall a_1,b_1,a_2,b_2,k \in \Z, k\neq 0,1,-1$. \\
Si $a_1 \equiv a_2 (mod\;k)$ et $b_1 \equiv b_2 (mod\;k)$, alors $\left\{\begin{matrix}
					a_1 + b_1 \equiv a_2 + b_2 (mod\;k)\\ 
					a_1 b_1 \equiv a_2 b_2 (mod\;k)
					\end{matrix}\right.$\\
En conséquence: $\forall c\in \Z: a_1 c \equiv a_2 c(mod\;k)$
\end{enumerate}

\begin{exmp}
$6 \equiv 2(mod\;4)$\\
$7 \equiv 0(mod\;7)$
\end{exmp}

\subsection{Cryptologie: Le système RSA}

Pour comprendre le système de cryptage RSA, on aura besoin d'un résultat technique.

\begin{lemme}
$\forall z \in \N: (z+1)^p \equiv n^p + 1(mod\;p)$ si p est un nombre premier.
\end{lemme}

\begin{thrm}[Le petit théorème de Fermat]
Soit $p\in \N$ un nombre premier.\\
Soit $a\in \N$ un nombre tel que $p\cancel{\;|\;}a$ (p ne divise pas a).\\
Alors, $a^{p-1} \equiv 1(mod\;p)$\\
\end{thrm}

\begin{demo}
Nous allons procéder par plusieurs étapes.\\

\begin{enumerate}
	\item Montrons par récurrence que $\forall a \in \N: a^p \equiv a(mod\;p)$.\\
	a = 1 : $1^p = 1(mod\;p)$\\
	Supposons vrai pour $a \in \N$ et montrons pour $a+1$.\\
	Par le lemme, on sait que $(a+1)^p \equiv a^p + 1(mod\;p)$. Alors, par hypothèse de récurrence: $(a+1)^p \equiv a + 1(mod\;p)$.\\
	\item On va maintenant utiliser $p\cancel{\;|\;}a$.\\
	On a: $\forall a \in \N: a^p \equiv a(mod\;p)$.\\
	$\Rightarrow$ Dans $\Z/p\Z: \overline{ap}=\overline{a}$ et comme $p\cancel{\;|\;}a: \exists \overline{b} \in \Z/p\Z$ un inverse de $\overline{a}$.\\
	$\Rightarrow \overline{b}\overline{a^p} = \overline{b}\overline{a}$\\
	$\Rightarrow \overline{b}\overline{a}^p = \overline{1}$\\
	$\Rightarrow \overline{a}^{p-1} = \overline{1} \Leftrightarrow a^{p-1} \equiv 1(mod\;p)$\\ 

\end{enumerate}
\end{demo}

\begin{demo} [du Lemme]
<WHOLE DEMO>
\end{demo}

\subsubsection{Fonctionnement des clés de chiffrement RSA}

2 personnes (A et B) veulent communiquer de manière sûre entre elles.\\

A choisit 2 nombres premiers p et q $\in \N$ appelés clé privée. A calcule:\\

\begin{enumerate}
\item $N = pq$
\item $O(N) = (p-1)(q-1)$
\item $e \in \Z$ tel que pgcd(e, O(N)) = 1
\end{enumerate}

appelé l'exposant de chiffrement.\\

O(N) et e sont premiers entre eux $\Rightarrow \exists 0 < s < O(N): es \equiv 1(mod\;O(N))$, c'est à dire que $\overline{s}$ est l'inverse de $\overline{e}$ dans $\Z/O(N)\Z$. \textbf{s est gardé secret}. \\

A publie les nombres (N,e) appelés la clé publique.\\

B souhaite envoyer un message à A. Dans le système RSA, la taille du message est $0<M<N$.\\

B utilise la clé publique et envoie le message chiffré: $\tilde{M} = M^e (mod\;N)$\\

Pour déchiffrer le message, A utilise s et obtient: $\tilde{M}^s = M^{es}(mod\;N) = M(mod\;N)$ (par le théorème suivant)\\