
\section{Arithmétique Modulaire}

\subsection{Les entiers et la division euclidienne}

L'ensemble des entiers est noté $\Z$, il contient les entiers naturels ($\N$) et leur opposé. Il est naturellement muni de 2 opérations qui satisfont les propriétés suivantes:

\begin{enumerate}
\item \textbf{L'addition} +: $\Z \times \Z \rightarrow \Z : (a,b) \rightarrow a+b $\\
Propriétés: 
	\begin{enumerate}
		\item \textbf{Associativité} $(a+b)+c = a+(b+c)$, $\forall a,b,c \in \Z$
		\item \textbf{Élément neutre} $0 \in \Z$: $a+0=a=0+a$, $\forall a \in \Z$
		\item \textbf{Opposé} $\forall a \in \Z: \exists -a \in \Z$ tel que $a+(-a) =0=(-a)+a$
		\item \textbf{Commutativité} $\forall a,b \in \Z: a+b = b+a$	\\
	\end{enumerate}
On dit que $(\Z,+)$ est un groupe (a,b,c) commutatif (d).\\

\item \textbf{La multiplication} $\cdot$: $\Z \times \Z \rightarrow \Z : (a,b) \rightarrow a \cdot b$ \\
Propriétés:
	\begin{enumerate}
		\item \textbf{Associativité} $a\cdot(b\cdot c)=(a\cdot b)\cdot c$
		\item \textbf{Distributivité par rapport à l'addition} \\
			\begin{minipage}{.2\textwidth}
				\hspace{0.5cm}$a\cdot (b+c) = ab + ac$\\

				\hspace{0.5cm}$(a+b)\cdot c = ac + bc$\\
			\end{minipage}
			\begin{minipage}{.2\textwidth}
				\hspace{1cm}$\forall a,b,c \in \Z$\\
			\end{minipage}
		\item \textbf{Commutativité} $a \cdot b = b\cdot a, \forall a,b \in \Z$
		\item \textbf{Élément neutre} $1 \in \Z$: $1 \cdot a=a=a \cdot 1$, $\forall a \in \Z$
		\item $\forall a,b,c \in \Z: a\cdot c = a \cdot b \Rightarrow c=b$ \\
	\end{enumerate}
On dit que $(\Z,+,\cdot)$ est un anneau ($(\Z,+)$ est un groupe commutatif et $\cdot$ satisfait (a) et (b) ) unital (d), commutatif (c) et intègre (e).\\
\end{enumerate}

On a aussi sur $\Z$ une relation d'ordre $\leq$ telle que:

\begin{enumerate}
	\item $\leq$ est un ordre total
	\item $\forall a,b,c \in \Z$, $a\leq b \Rightarrow a+c \leq b+c$
	\item $\forall a,b,c \in \Z$, $a\leq b$, $c\geq 0 \Rightarrow ac\leq bc$\\
\end{enumerate}

La valeur absolue est une application\\

\[ |\;|: \Z \rightarrow \N : a \rightarrow
  \begin{cases}
    a   & \quad \text{si } a\geq 0 \\
    -a  & \quad \text{si } a\leq 0 \\
  \end{cases}
\]

telle que: 

	\begin{enumerate}
		\item $\forall a \in \Z: |a| = 0$ ssi $a=0$
		\item $\forall a,b \in \Z: |a\cdot b| = |a| \cdot |b|$\\
	\end{enumerate}

Remarque: L'équation $ax = b, a,b\in \Z$ n'a pas toujours de solution dans $\Z$.\\

\begin{defn}
Soit $a,b \in \Z$, on dit que a divise b, et on note $a|b$, si $\exists c \in \Z$ tel que $ac=b$. On dit aussi que b est un multiple de a.
\end{defn}

\begin{prop}
| est une relation:

\begin{enumerate}
	\item \textbf{Réflexive} $\forall a \in \Z$: $a|a$
	\item \textbf{Transitive} $\forall a,b,c \in \Z$: $a|b$ et $b|c \Rightarrow a|c$
	\item \textbf{Anti-symétrique} $\forall a,b \in \Z$: $a|b$ et $b|a \Rightarrow a=\pm b$
\end{enumerate}
\end{prop}

\begin{thrm}[Division Euclidienne]
$\forall a,b \in \Z$, $b \neq 0$, $\exists$ des entiers uniques q (quotient) et r (reste) tels que $a= bq + r$ et $0 \leq r < |b|$
\end{thrm}

\begin{defn}
Un nombre $p \in \Z$ est premier si $p \neq 1, -1, 0$ et p n'est divisible que par 1, -1, p et -p.
\end{defn}

\begin{defn}
Un entier d est un plus grand commun diviseur (pgcd) de 2 entiers a et b ssi:
\begin{enumerate}
\item d|a, d|b
\item Si $c \in \Z: c|a$ et $c|b \Rightarrow c|d$
\end{enumerate}

On note pgcd(a,b) le pgcd positif de a et b $\in \Z_0$ et on pose pgcd(a,0) = |a|, $\forall a \in \Z_0$
\end{defn}

\subsubsection{L'algorithme d'Euclide}

\begin{prop}
Si $a, b \in \Z, b \neq 0$ et $q,r \in \Z$ tel que $a=bq + r$, alors pgcd(a,b) = pgcd(b,r).
\end{prop}

\begin{demo}
Comme a = bq + r, si c|b et r $\Rightarrow$ c|a.
Comme a - bq = r, si c|a et b $\Rightarrow$ c|r.
Donc, l'ensemble des diviseurs communs de b et r est égal à l'ensemble des diviseurs communs de a et b.
\end{demo}

\hspace{-0.55cm}\textbf{Algorithme d'Euclide:} Pour trouver explicitement pgcd(a,b) $\forall a,b \in \Z, b\neq 0$, on procède comme suit:

On suppose que a et b $\geq 0$ car pgcd(a,b) = pgcd(-a,b) = pgcd(a,-b) = pgcd(-a,-b).

Par le théorème de division euclidienne: $a = bq_1 + r_1$ pour $q_1 \in \Z$ et $0 \leq r_1 < b$ alors $pgcd(a,b) = pgcd(b,r_1)$. 

Si $r_1 = 0$ alors pgcd(a,b) = pgcd(b,0) = b. Sinon, $r_1 > 0$ et par division euclidienne: $b = r_1 q_2 + r_2 \;\; 0 \leq r_2 < r_1$.

Si $r_2 = 0 \Rightarrow pgcd(a,b) = r_1$. Sinon, on itère $r_1 = r_2 q_3 + r_3 \;\; 0 \leq r_3 < r_2$.

On obtient des restes $r_1,r_2,\ldots$ qui sont des entiers non négatifs décroissants. Il doit donc exister $n \in \N_0$ tel que $r_{n+1} = 0$ et $r_n \neq 0$.

$pgcd(a,b) = pgcd(b,r_1) = pgcd(r_1,r_2) = \ldots = pgcd(r_{n-1},r_n) = pgcd(r_n,0) = r_n$

\newpage

\begin{prop}[\textcolor{red}{Identité de Bézout}]
Soit $a,b \in \Z$, alors $\exists s,t \in \Z$ tels que pgcd(a,b) = sa + tb.
\end{prop}

\begin{demo}
Supposons que a et b > 0. Soient $r_1,\ldots ,r_n$ les restes de la division euclidienne donnés par l'algorithme d'Euclide tels que:

\begin{minipage}{.4\textwidth}
	$a = bq_1 + r_1$\\
	$b = r_1 q_2 + r_2$\\
	$r_i = r_{i+1} q_{i+2}+r_{i+2}$\\
	$r_{n-3} = r_{n-2} q_{n-1}+r_{n-1}$\\
	$r_{n-2} = r_{n-1} q_{n}+r_{n}$\\
	$r_{n-1} = r_{n} q_{n+1}+0$
\end{minipage}
\begin{minipage}{.4\textwidth}
	$0 < r_1 < b$\\
	$0 < r_2 < r_1$\\
	$0 < r_{i+2} < r_{i+1}$\\
	$0 < r_n < r_{n-1}$
\end{minipage}

\hspace{-0.5cm}$pgcd(a,b) = pgcd(b,r_1) = pgcd(r_2,r_3) = \ldots = pgcd(r_{n-1},r_n)$

\hspace{-0.5cm}On pose $r_0 = b$ et $r_{-1} = a$ pour l'algo.

\hspace{-0.5cm}On construit s et t comme suit:

\[
pgcd(a,b) = r_n = r_{n_2} - q_n r_{n-1} = s_1 r_{n-2} + t_1 r_{n-1} \quad \text{avec} \;\; s_1 =1 , t_1 = -q_n
\]

\hspace{-0.5cm}Ensuite, on utilise $r_{n-3} = r_{n-2} q_{n-1} + r_{n-1} \Leftrightarrow r_{n-1} = r_{n-3} - q_{n-1} r_{n-2}$

\begin{align*}
\text{Donc,} \;\; pgcd(a,b) & = s_1 r_{n-2} + t_1 ( r_{n-3} - q_{n-1} r_{n-2} ) = t_1 r_{n-3} + (s_1 -q_{n-1} t_1) r_{n-2} \\
 & = s_2 r_{n-3} + t_2 r_{n-2} \quad \text{avec} \;\; s_2 = t_1 , t_2 = s_1 - q_{n-2} t_1
\end{align*}

\hspace{-0.5cm}Inductivement, on construit $s_k = t_{k-1}$ et $t_k = s_{k-1} - t_{k-1} q_{n-(k-1)}$ tel que pgcd(a,b) = $s_k r_{n-(k+1)} + t_k r_{n-k}$

\hspace{-0.5cm}$\Rightarrow pgcd(a,b) = pgcd(r_{-1},r_0) = s_n r_{-1} + t_n r_0 = s_n a + t_n b$ avec $s_n = t_{n-1}$ et $t_n = s_{n-1}-t_{n-1} q_1$

\end{demo}

\subsection{Groupes, anneaux et entiers modulo n}

\subsubsection{Définitions}

Les groupes apparaissent naturellement dès qu'on parle des symétries d'un objet. 

\begin{defn}
Un groupe $(G,*)$ est un ensemble non vide G muni d'une loin de composition $* : G \times G \rightarrow G$ tel que:

\begin{enumerate}[(i)]
\item * soit associative
\item $\exists e \in G: g*e = g = e*g$ (e est appelé le neutre)
\item $\forall g \in G: \exists g^{-1} \in G$ tq $g*g^{-1} = e = g^{-1}*g$
\end{enumerate}
\end{defn}

\begin{exmp} $\;$
	\begin{enumerate}
	\item $(\Z,+)$ est un groupe
	\item $(\Z_0,\cdot)$ n'est pas un groupe
	\item $(\R,\cdot)$ n'est pas un groupe
	\item $(\R_0,\cdot)$ est un groupe
	\end{enumerate}
\end{exmp}

\begin{defn}
Soit (G,*) un groupe. Un sous-ensemble $H \subseteq G$ est un sous-groupe de G si (H,*) est un groupe. On note $H \leq G$ ou $(H,*) \leq (G,*)$
\end{defn}

\newpage

\begin{prop}
Soit (G,*) un groupe. $H \subseteq G$ est un sous-groupe de G ssi:
	\begin{enumerate}
	\item $e \in H$
	\item $\forall g,h \in H: g*h^{-1} \in H$
	\end{enumerate} 
\end{prop}

\begin{exmp}
$2\Z = \{ \ldots, -2, 0, 2, \ldots \} = \{ 2z | z \in \Z \}$ est un sous-groupe de $(\Z,+)$\\

\hspace{-0.5cm}$3\Z = \{ \ldots, -3, 0, 3, \ldots \} = \{ 3z | z \in \Z \}$ est un sous-groupe de $(\Z,+)$
\end{exmp}

\begin{prop}
Soit $S \subseteq \Z$ un sous-ensemble non-vide tel que $(S,+) \leq (\Z,+) \Rightarrow \; \exists \; k \in \Z$ tel que $S = k\Z$
\end{prop}

\begin{demo}
Posons k le plus petit entier positif dans S $\Rightarrow k\Z \subseteq S$

\hspace{-0.5cm}Supposons que $\exists s \in S: s \notin k\Z$ (on peut supposer s > 0)

\hspace{-0.5cm}$\overset{Div Eucl}{\Rightarrow} s=kq + r$ avec $0 < r < k$

\hspace{-0.5cm}$\Rightarrow r = s-kq \in S \Rightarrow$ Incompatible avec le fait que k est le plus petit entier.
\end{demo}

\begin{exmp}
Interprétation du pgcd: Soit $k, l \in \Z$. On définit $k\Z + l\Z = \{ kz_1 + lz_2 | z_1, z_2 \in \Z \}$. On a $k\Z + l\Z = pgcd(k,l)\Z$.
\end{exmp}

\begin{demo} $\;$
\begin{enumerate}[1)]
	\item Montrer que $pgcd(k,l)\Z \subseteq k\Z + l\Z$: On sait que $\exists s,t$ tel que $pgcd(k,l) = ks + lt \in k\Z + l\Z$.
	\item Montrer que $k\Z + l\Z \subseteq pgcd(k,l)$: Soit $z_1,z_2 \in \Z$ et $kz_1 + lz_2 \in k\Z + l\Z$. Par définition du pgcd: $\exists y_1,y_2$ tel que $k=pgcd(k,l)y_1, \;\; l=pgcd(k,l)y_2 \Rightarrow lz_1 + lz_2 = pgcd(k,l)(y_1 z_1 + y_2 z_2) \in pgcd(k,l)\Z$ 
\end{enumerate}
\end{demo}

\subsubsection{Groupes quotients}

Soit (G,+) un groupe commutatif. Dans ce cas, on note l'inverse de g, $\overline{g}$.

\begin{defn}
Soit $(N,+) \leq (G,+)$. Une classe latérale de N est un ensemble $g + N : \{ g+n | n \in N \}$ pour un élément fixé $g \in G$.
\end{defn}

\begin{prop}
Deux éléments $g, {g}' \in G$ dét. la même classe latérale de N ssi $g+N = {g}' + N \Leftrightarrow \forall n \in N, \exists! {n}' \in N$ tel que $g+n={g}' + {n}'$
\end{prop}

\begin{demo}
$\Leftarrow$ OK

\hspace{-0.55cm}$\Rightarrow$
	\begin{enumerate}[$\cdot$]
		\item $\forall n \in N: g+n \in g + N = {g}' + N \Rightarrow \exists {n}' \in N$ tq $g+n = {g}' + {n}'$
		\item Unicité: ${n}'$ et ${n}''$ tels que ${g}' + {n}' = {g}' + {n}'' \Rightarrow {g}' + (-{g}') = {n}'' + (-{n}') \Rightarrow e={n}'' + (-{n}') \Leftrightarrow {n}' = {n}''$
	\end{enumerate}
\end{demo}

\begin{defn}
On note G/N l'ensemble de classe latérale de N. G/N = $\{ g + N | g \in G \}$
\end{defn}

\begin{exmp}
$(\Z,+)$ et $7 \in \Z: \Z/7\Z = \{ 0+7\Z, 1+7\Z, \ldots, 6+7\Z \}$ que l'on note $\{ \overline{0}, \overline{1},\ldots, \overline{6}\}$. Remarque: $\overline{7} = \overline{0}$
\end{exmp}

\newpage

\begin{prop}
Soit le groupe $(\Z,+)$ et $k \in \Z$. Alors $\Z/k\Z$ est une partition de $\Z$.
\end{prop}

\begin{demo} $\;$
\begin{enumerate}[$\cdot$]
	\item $z \in \Z \Rightarrow \exists q,r$ tq $z = kq + r \in r+k\Z$
	\item $r_1,r_2 \in \Z$: montrer que $r_1 + k\Z \cap r_2 + k\Z \neq 0 \Leftrightarrow r_1 + k\Z = r_2 + k\Z \Leftrightarrow r_1 + k q_1 = r_2 + k q_2 \Leftrightarrow r_1 - r_2 = k(q_1-q_2) \in k\Z \Leftrightarrow r_1 + k\Z = r_2 + k\Z$  
\end{enumerate}
\end{demo}

\begin{thrm}
Soit (G,+) un groupe commutatif et $N \leq G$. Alors G/N est muni d'une loi $\overline{+}$ tel que $(G/N,\overline{+})$ soit un groupe commutatif. Précisément, on définit \[ \forall g,{g}' \in G: \overline{g} \overline{+} \overline{{g}'} = \overline{g+{g}'} = (g+N) \overline{+} ({g}' + N) = g + {g}' + N \]
\end{thrm}

\begin{demo} $\;$
\begin{enumerate}[$\cdot$]
	\item On montre que $\overline{+}$ est bien défini. C'est à dire: $g, {g}'$ et $h, {h}' \in G$ tels que $\overline{g} = \overline{h}$ et $\overline{{g}'} = \overline{{h}'}$. Montrons que $\overline{g+{g}'} = \overline{h+{h}'}$.

	\hspace{1cm} $ \overline{g} = \overline{h} \Rightarrow g-h=n, \quad n\in N $

	\hspace{1cm} $ \overline{{g}'} = \overline{{h}'} \Rightarrow {g}' - {h}' = {n}' , \quad {n}' \in N$

	\hspace{1cm} Donc, $g+{g}' - (h+{h}') = g + {g}' + (-h) + (-{h}') = (g-h) + ({g}' + {h}') = n + {n}' \in N$

	\hspace{1cm} $\Rightarrow \overline{g+{g}'} = \overline{h+{h}'}$
	\item $\overline{+}$ est associative car + l'est: $\forall g,{g}',{g}''$

	\hspace{1cm} $(\overline{g}\;\overline{+}\;\overline{{g}'})\;\overline{+}\;\overline{{g}''} = \overline{g + {g}'}\;\overline{+}\;\overline{{g}''} = \overline{(g+{g}')+{g}''}$

	\hspace{1cm} $\overline{g}\;\overline{+}\;(\overline{{g}'}\;\overline{+}\;\overline{{g}''}) = \overline{g}\;\overline{+}\;\overline{{g}' + {g}''} = \overline{g+({g}' + {g}'')}$

	\item $\overline{+}$ est commutative

	\item le neutre est $\overline{e}$ où e est le neutre de (G,+)

	\item L'inverse/opposé de $\overline{g}$ est $\overline{-g}$
\end{enumerate}
\end{demo}

\begin{defn}[et exemple principal]
Pour $n\in \N$, on définit le groupe des entiers modulo n comme le groupe quotient $(\Z/n\Z,+)$ où $\overline{a} + \overline{b} = \overline{a+b} \qquad \forall a,b \in \Z$
\end{defn}

\begin{exmp}
$\Z/8\Z = \{ \overline{0}, \overline{1}, \ldots, \overline{7} \}$.

\hspace{-0.55cm}$\overline{0}$ est neutre.

\hspace{-0.55cm}$\overline{2} + \overline{3} = \overline{5}$

\hspace{-0.55cm}$\overline{6} + \overline{10} = \overline{16} = \overline{0}$
\end{exmp}

\newpage

\subsubsection{Isomorphismes de groupe}

\begin{defn}
Soit $(G,*)$ et $({G}',*)$ deux groupes. Un morphisme de groupe est une application $f: G \rightarrow {G}'$ tel que $\forall g,h \in G: f(g*h) = f(g) * f(h)$
\end{defn}

\begin{exmp}
\textbf{1)} $(\R,+)$ et $(\R^+_0,\cdot) \qquad$ $exp: \begin{matrix} \R \rightarrow \R^+_0\\ x \rightarrow e^x \end{matrix}$ \\

$morphisme \; \; e^{x+y} = e^x \cdot e^y$ \\

\hspace{-0.55cm}\textbf{2)}  $(\R^+_0,\cdot)$ et $(\R,+) \qquad$ $exp: \begin{matrix} \R^+_0 \rightarrow \R \\ x \rightarrow log(x) \end{matrix}$ \\

$morphisme \; \; log(x\cdot y) = log(x) + log(y)$ \\
\end{exmp}

\begin{defn}
Un morphisme de groupe f: $G \rightarrow {G}'$ est dit:
	\begin{description}
	\item[Injectif:] Si $\forall g_1,g_2 \in G: f(g_1)=f(g_2) \Rightarrow g_1 = g_2$
	\item[Surjectif:] Si $\forall {g}' \in G: \exists g\in G$ tq $f(g) = {g}'$
	\end{description}
\end{defn}

\begin{defn}
Soit (G,*) et $({G}',{*}')$ deux groupes et f: $G \rightarrow {G}'$ un morphisme de groupe.

	\begin{enumerate}[-]
	\item L'image de f est l'ensemble Im(f) = $\{ {g}' \in {G}' | \exists g \in G : f(g) = {g}' \} \subseteq {G}'$
	\item Le noyau de f est l'ensemble Ker(f)= $\{ g\in G | f(g) = e \} \subseteq G$
	\end{enumerate}
\end{defn}

\begin{prop}
Soient (G,*) et $({G}',{*}')$ deux groupes et f: $G \rightarrow {G}'$ un morphisme de groupe. Alors:
	\begin{enumerate}
	\item f est injectif $\Leftrightarrow$ Ker(f) = $\{ e \}$
	\item f est surjectif $\Leftrightarrow$ Im(f) = ${G}'$
	\end{enumerate}
\end{prop}

\begin{defn}
Soient (G,*) et $({G}',{*}')$ deux groupes. Un isomorphisme de groupe est un morphisme bijectif f: $G \rightarrow {G}'$. Ils sont dits isomorphes et on note $(G,*) \cong ({G}',{*}')$
\end{defn}

\begin{thrm}[Théorème d'isomorphisme]
Soient (G,*) et $({G}',{*}')$ deux groupes commutatifs et f un morphisme de groupe surjectif: f: $G \rightarrow {G}'$, Im(f) = ${G}' \Rightarrow G/Ker(f) \cong {G}'$ 
\end{thrm}

\textbf{L'ordre d'un groupe et de ses éléments}

\begin{defn}
Soit (G,*) un groupe fini. L'ordre de G est le cardinal de G. On le note $\#G$.
\end{defn}

\begin{prop}
Soit (G,*) un groupe et $g\in G, g \neq e \Rightarrow <g> = \{ g^k | k \in \Z \}$ est un sous-groupe commutatif de G où:
	\begin{enumerate}
		\item $g^0 = e$
		\item $g^k = g*\ldots*g \qquad si\; k\in \N_0$
		\item $g^{-k} = (g^{-1})^k \qquad si\; k\in \N_0$
	\end{enumerate}
\end{prop}

\begin{prop}
Soit (G,*) un groupe fini et $e \neq g \in G \Rightarrow \exists k \in \N_0: g^k = e$
\end{prop}

\begin{defn}
Soig (G,*) un groupe et $g\in G$. On définit l'ordre de g et on note O(g) le plus petit nombre naturel non nul tel que $g^{O(g)}=e$. Si $g^k \neq e, \forall k:$ on pose $O(g) = \infty$
\end{defn}

\subsubsection{Les anneaux}

\begin{defn}
Un anneau $(A,+,\cdot)$ est un ensemble non vide A muni de 2 lois de composition: $+ : A \times A \rightarrow A \;\; (a,b) \rightarrow a+b$ et $\cdot : A \times A \rightarrow \;\; (a,b) \rightarrow a\cdot b$

\begin{enumerate}
	\item (A,+) est un groupe commutatif
	\item La multiplication est associative: (ab)c = a(bc), $\forall a,b,c \in A$.
	\item La multiplication est distributive: (a+b)c = ac + bc et a(b+c)= ab + ac, $\forall a,b,c \in A$.
\end{enumerate}
\end{defn}

\begin{defn}
Lorsque $\cdot$ est commutative: $(A,+,\cdot)$ est un anneau commutatif.

Si $\exists 1 \in A$ tq $a\cdot 1 = a = 1 \cdot a \quad \forall a \neq 0 \in A \Rightarrow (A,+,\cdot)$ est un anneau unital.
\end{defn}

\begin{exmp} $\;$
	\begin{enumerate}
	\item $(\Z,+,\cdot)$ anneau commutatif unital.
	\item $(M_2 (\R) = { \bigl( \begin{smallmatrix} a b&\\c d& \end{smallmatrix}\bigr)  | a,b,c,d \in \R },+,\cdot )$ anneau unital (non commutatif) 
	\end{enumerate}
\end{exmp}

\begin{prop}
Soit $k \neq 1 \in \Z_0$ et soit le groupe $(\Z/k\Z,\overline{+})$. On définit $\overline{\cdot}: \Z/k\Z \times \Z/k\Z \rightarrow \Z/k\Z$ tel que $\forall l,{l}' \in \Z: \overline{l}\; \overline{\cdot}\; \overline{{l}'} = \overline{l\;{l}'}$.

\hspace{-0.55cm}Alors, $(\Z/k\Z,\overline{+},\overline{\cdot})$ est un anneau commutatif unital.
\end{prop}

\begin{demo}
Il faut montrer que $\overline{\cdot}$ est bien défini.

\[ l_1 \;et\; l_2 \;tq\; \overline{l_1} = \overline{l_2} \Rightarrow l_1 = l_2 + kz  \]

\[ {l_1}' \;et\; {l_2}' \;tq\; \overline{{l_1}'} = \overline{{l_2}'} \Rightarrow {l_1}' = {l_2}' + k{z}' \]

\[\Rightarrow l_2 \cdot {l_2}' = (l_1 - kz)({l_1}' - k{z}') = l_1 {l_1}' - k{l_1}'z - kl_1{z}' - kkz\overline{z} \]

\[ \Rightarrow \overline{l_2 {l_2}'} = \overline{l_1 {l_1}'} \]

Les autres propriétés découlent des propriétés de + et $\cdot$ dans $\Z$.

Remarque: $\overline{1}$ est neutre pour $\overline{\cdot}$
\end{demo}

\begin{exmp}
Dans $(\Z/10\Z, \overline{+}, \overline{\cdot})$: 

\hspace{-0.55cm}$\overline{1}\; \overline{\cdot}\; \overline{2} = \overline{2}$

\hspace{-0.55cm}$\overline{2}\; \overline{\cdot}\; \overline{5} = \overline{10} = \overline{0}$
\end{exmp}

\newpage

\textbf{Interprétation des pgcd, nombres premiers:}

\begin{defn}
Soit $(A,+,\cdot)$ un anneau:
	\begin{enumerate}
		\item $a \in A$ est inversible si $\exists b \in A: ab=1$
		\item $a \in A$ est un diviseur de 0 si $\exists b \in A, b \neq 0 : ab = 0$
	\end{enumerate}
\end{defn}

\begin{exmp}
Soient $0 < a \leq b < k \in \N_0$ tels que $ab = k$

\hspace{-0.55cm}$\Rightarrow \overline{a}$ et $\overline{b}$ sont des div. de zéro dans $\Z/k\Z$
\end{exmp}

\begin{prop}[\textcolor{red}{ATTENTION! Cette proposition fait partie de ceux à connaitre par coeur à l'examen! (pour l'année 2015-2016)}]
Soit $k \in \N_0, k > 1$ et soit $l \in \Z$ alors: $\overline{l}$ est inversible dans $\Z/k\Z \Leftrightarrow$ l et k sont premiers entre eux.
\end{prop}

\begin{demo}
$\Leftarrow$ : pgcd(l,k) = 1 $\Rightarrow \exists s,t \in \Z: sl + tk = 1$. Donc, $\overline{sl+tk} = \overline{1} \Rightarrow \overline{sl} = \overline{1} \Rightarrow \overline{s}\; \cdot \; \overline{l} = \overline{1}$ 

\hspace{-0.55cm}$\Rightarrow$: $\exists s \in \Z$ tel que $\overline{l} \; \cdot \; \overline{s} = \overline{1} \Leftrightarrow \exists t \in \Z: ls= 1 + kt \Rightarrow sl + (-t)k = 1 \overset{Bezout}{\Rightarrow}$ k et l sont premiers entre eux.
\end{demo}

\begin{exmp}
Dans $\Z/5\Z: \overline{2}$ est inversible $(\overline{2}\;\cdot\;\overline{3} = \overline{1})$

\hspace{-0.55cm}Dans $\Z/6\Z: \overline{2}$ n'est inversible et $(\overline{2}\;\cdot\;\overline{3} = \overline{0})$
\end{exmp}

\begin{defn}
$(K,+,\cdot)$ est un champ si $(K,+,\cdot)$ est un anneau commutatif unital et si tout élément non nul de K admet un inverse pour la multiplication.
\end{defn}

\begin{prop}
Soit $k \in \N_0, k>1,$ alors: $\Z/k\Z$ est un champ ssi k est un nombre premier.
\end{prop}

\begin{demo}
(SUPER LONG)
\end{demo}

\subsubsection{Relation de congruence}

\begin{defn}
Soient $a,b,k \in \Z, k \neq 0,1,-1$. On dit que a est congru à b modulo k et on note $a \equiv b(mod\;k)$ si $a-b \in k\Z$ (ou encore si $\overline{a} = \overline{b}$ dans $\Z/k\Z$). \\
\end{defn}

Propriétés:

\begin{enumerate}
\item La congruence modulo k est une relation d'équivalence.
	\begin{enumerate}[-]
	\item \textbf{Réflexivité} $\forall a \in \Z: a \equiv a(mod\;k)$
	\item \textbf{Symétrie} $\forall a,b \in \Z: a \equiv b(mod\;k) \Leftrightarrow b \equiv a(mod\;k)$
	\item \textbf{Transitivité} $\forall a,b,c \in \Z:$
				$ \left\{\begin{matrix}
				a \equiv b(mod\;k)\\ 
				b \equiv c(mod\;k)
				\end{matrix}\right.$ $\quad \Rightarrow a\equiv c(mod\;k)$
	\end{enumerate}

\item $\forall a_1,b_1,a_2,b_2,k \in \Z, k\neq 0,1,-1$. \\
Si $a_1 \equiv a_2 (mod\;k)$ et $b_1 \equiv b_2 (mod\;k)$, alors $\left\{\begin{matrix}
					a_1 + b_1 \equiv a_2 + b_2 (mod\;k)\\ 
					a_1 b_1 \equiv a_2 b_2 (mod\;k)
					\end{matrix}\right.$\\
En conséquence: $\forall c\in \Z: a_1 c \equiv a_2 c(mod\;k)$
\end{enumerate}

\begin{exmp}
$6 \equiv 2(mod\;4)$\\
$7 \equiv 0(mod\;7)$
\end{exmp}

\newpage

\subsection{Cryptologie: Le système RSA}

Pour comprendre le système de cryptage RSA, on aura besoin d'un résultat technique.

\begin{lemme}
$\forall n \in \N: (n+1)^p \equiv n^p + 1(mod\;p)$ si p est un nombre premier.
\end{lemme}

\begin{thrm}[Le petit théorème de Fermat (\textcolor{red}{ATTENTION! Ce théorème, le lemme le précédent, et leurs démonstrations mutuelles font partie de ceux à connaitre par coeur à l'examen! (pour l'année 2015-2016)})]
Soit $p\in \N$ un nombre premier.\\
Soit $a\in \N$ un nombre tel que $p\cancel{\;|\;}a$ (p ne divise pas a).\\
Alors, $a^{p-1} \equiv 1(mod\;p)$\\
\end{thrm}

\begin{demo}
Nous allons procéder par plusieurs étapes.\\

\begin{enumerate}
	\item Montrons par récurrence que $\forall a \in \N: a^p \equiv a(mod\;p)$.\\
	a = 1 : $1^p = 1(mod\;p)$\\
	Supposons vrai pour $a \in \N$ et montrons pour $a+1$.\\
	Par le lemme, on sait que $(a+1)^p \equiv a^p + 1(mod\;p)$. Alors, par hypothèse de récurrence: $(a+1)^p \equiv a + 1(mod\;p)$.\\
	\item On va maintenant utiliser $p\cancel{\;|\;}a$.\\
	On a: $\forall a \in \N: a^p \equiv a(mod\;p)$.\\
	$\Rightarrow$ Dans $\Z/p\Z: \overline{ap}=\overline{a}$ et comme $p\cancel{\;|\;}a: \exists \overline{b} \in \Z/p\Z$ un inverse de $\overline{a}$.\\
	$\Rightarrow \overline{b}\overline{a^p} = \overline{b}\overline{a}$\\
	$\Rightarrow \overline{b}\overline{a}^p = \overline{1}$\\
	$\Rightarrow \overline{a}^{p-1} = \overline{1} \Leftrightarrow a^{p-1} \equiv 1(mod\;p)$\\ 

\end{enumerate}
\end{demo}

\begin{demo}
\[
(n+1)^p = \sum_{i=0}^p {p \choose i} n^i = n^p + \sum_{i=0}^{p-1} {p \choose i} n^i + 1 = n^p + 1 + \sum_{i=1}^{p-1} {p \choose i} n^i (mod \; p)
\]

Par récurrence sur i: on va montrer que $p|{p \choose i}$ pour $i=1,\ldots,(p-1)$. i=1: ${p \choose 1} = p$ et $p|p$

Supposons que $p|{p \choose i}$ pour $1 \leq i < p-1$ et montrons que $p|{p \choose i+1}$

\[ {p \choose i+1} = \frac{p!}{(i+1)!(p-i-1)} = {p \choose i} \frac{(p-i)}{(i+1)} \]

\[ p|{p \choose i} \Rightarrow {p \choose i} = pb \text{\;pour\;} b\in\Z \Rightarrow {p \choose i+1} = \frac{pb(p-i)}{(i+1)} \in \N \]

p premier et $1 < i+1 \leq p+1$

\[ \Rightarrow (i+1)|b(p-i) \qquad (\text{\;car \;}(i+1)\cancel{\;|\;}p) \]

\[ \Rightarrow {p \choose i+1} = p\cdot \frac{b(p-i)}{i+1} \Rightarrow p|{p \choose i+1} \]

\[ \Rightarrow {p \choose i} = 0(mod\; p) \qquad \text{\;pour \;} i=1,\ldots,p-1 \]

\[ \Rightarrow (n+1)^p = n^p + 1(mod\; p) \]

\end{demo}

\subsubsection{Fonctionnement des clés de chiffrement RSA}

2 personnes (A et B) veulent communiquer de manière sûre entre elles.

A choisit 2 nombres premiers p et q $\in \N$, $p \neq q$, appelés clé privée. A calcule:

\begin{enumerate}
\item $N = pq$
\item $O(N) = (p-1)(q-1)$
\item $e \in \Z$ tel que pgcd(e, O(N)) = 1
\end{enumerate}

appelé l'exposant de chiffrement.

O(N) et e sont premiers entre eux $\Rightarrow \exists 0 < s < O(N): es \equiv 1(mod\;O(N))$, c'est à dire que $\overline{s}$ est l'inverse de $\overline{e}$ dans $\Z/O(N)\Z$. \textbf{s est gardé secret}. 

A publie les nombres (N,e) appelés la clé publique.

B souhaite envoyer un message à A. Dans le système RSA, la taille du message est $0<M<N$.

B utilise la clé publique et envoie le message chiffré: $\tilde{M} = M^e (mod\;N)$

Pour déchiffrer le message, A utilise s et obtient: $\tilde{M}^s = M^{es}(mod\;N) = M(mod\;N)$ (par le théorème suivant)

\begin{thrm}
$\forall 0 < M < N = pq \in \Z$, p et q premiers. Soit $u = 1 (O(N))$. Alors $M^u = M(mod \; N)$.
\end{thrm}

\begin{demo}
$O < M < N \Rightarrow p \cancel{\;|\;} M$ ou $q \cancel{\;|\;} M$

\begin{description}
	\item[Cas 1] $p \cancel{\;|\;} M$ et $q \cancel{\;|\;} M$
		$\quad u = 1 + t O(N) = 1 + t(p-1)(q-1)$
		$\Rightarrow M^u = M + M^{t(p-1)(q-1)}$
		Or, par le petit théorème de Fermat: 
			\[ (M^{t(p-1)})^{q-1} = 1 (mod\; q)\]
			\[ (M^{t(q-1)})^{p-1} = 1 (mod\; p)\]
			\[ \Rightarrow \left\{\begin{matrix}
				M^u = M (mod\; q) \\ 
				M^u = M (mod\; p)
				\end{matrix}\right. \Rightarrow \left\{\begin{matrix}
				M^u - M= 0(mod\;q)\\ 
				M^u - M= 0(mod\;p)
				\end{matrix}\right. \]
			\[ \text{ie}\; p|M^u - M \text{et}\; q|M^u - M \Rightarrow pq| M^u - M \]
			\[ \Rightarrow M^u - M = 0(mod\; N) \]
			\[ \Rightarrow M^u = M(mod \; N) \]
	\item[Cas 2] $ p | M$ et $ q \cancel{\;|\;} M$
			$\quad u = 1 + t(p-1)(q-1)$
			\[ q \cancel{\;|\;} M \xrightarrow[de\;\; Fermat]{Petit \;\;Thrm} (M^{t(p-1)})^{q-1} = 1(mod \; q) \]
			\[ \Rightarrow M^{t(p-1)(q-1)} = 1 + lq \text{\;pour\;un\;} l \in \Z \]
			\[ \text{Donc,\;} M^u = M M^{t(p-1)(q-1)} = M(1+lq) \]
			\[ p|M \Rightarrow \exists c \in \Z: pc = M \]
			\[ \Rightarrow M^u  = M(1+lq) = pc(1+lq) = pc + lpcq = M + lcpq \]
			\[ \Rightarrow M^u = M(mod \; N) \]
	\item[Cas 3] $p \cancel{\;|\;} M$ et $q | M$ (exo à faire chez soi)
\end{description}
\end{demo}