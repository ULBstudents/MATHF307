\documentclass[a4paper,10pt]{article}
\usepackage[a4paper]{geometry}
\geometry{hscale=0.8,vscale=0.8,centering}

\usepackage[utf8]{inputenc}
\usepackage[T1]{fontenc}
\usepackage[french]{babel} % Exposant

\usepackage{enumerate} % Listes
\usepackage{amsmath,amsfonts,amssymb,amsthm} % Matrices, Align
\usepackage{thmtools}
\usepackage{mathtools}
\usepackage{graphicx}
\usepackage{ulem}
\usepackage{listings} % Lecture du code
\usepackage{hyperref} % Hyperlien
\usepackage{color} % Couleurs
\usepackage[dvipsnames]{xcolor} % Couleurs dans les tableaux
\usepackage{titlesec}
\usepackage{subfigure}
\usepackage{xstring}
\usepackage{tkz-graph}
\usepackage{tkz-berge}
\usepackage{float}
\usepackage[makeroom]{cancel}

\usepackage{array} % Tableau pour les formules de math
\newcolumntype{L}{>{\displaystyle} l}
\newcolumntype{C}{>{\displaystyle} c}
\newcolumntype{R}{>{\displaystyle} r}
\def\arraystretch{1.5}

\usepackage{fancyhdr} % Mise en page spéciale fancyhdr pour les en-têtes
\pagestyle{fancy}

\usepackage{tikz}
\usetikzlibrary{graphs,positioning,calc,backgrounds,arrows,fit,shapes}

\setlength{\parskip}{0.5em}

\newcommand{\R}{\mathbb{R}} 
\newcommand{\Z}{\mathbb{Z}} 
\newcommand{\N}{\mathbb{N}}

\newcommand\Twocolor[4]{
\path 
	let 
	\p1=($ #1 - #2 $ ),
	\p2=#1,
	\p3=#2
	in 
	node[
		draw,
		outer sep=0pt,
		inner sep=0pt,
		minimum height=3pt,
		text width={veclen(\x1,\y1)},
		rotate={atan((\y3-\y2)/(\x3-\x2))},
		anchor=west
	] (mydouble) at #1 {};
	\draw[#3,line width=1pt] (mydouble.north west) -- (mydouble.north east); 
	\draw[#4,line width=1pt] (mydouble.south west) -- (mydouble.south east); 
}

\newcommand{\vmodels}{\rotatebox[origin=c]{90}{$\models$}}

\title{MATH-F307 - Mathématiques Discrètes\\Laurent \textsc{La Fuente}\\Notes de cours\\} % Titres
\author{André \textsc{Madeira Cortes}\\Nikita \textsc{Marchant}} % Auteurs depuis pdfinfo.tex
\date{ } % Date de compilation du pdf


\renewcommand{\headrulewidth}{0pt}
\fancyhead[C]{} % Rien en haut de page au milieu
\fancyhead[L]{\leftmark} % Nom du chapitre actuel en haut de page à gauche
\fancyhead[R]{\thepage} % Numéro de page en haut de page à droite

\renewcommand{\footrulewidth}{0pt}
\fancyfoot[C]{} % Rien en bas de page au milieu
\fancyfoot[L]{} % Lien de bas de page à gauche
\fancyfoot[R]{\thepage} % Numéro de page en bas de page à droite
\setlength{\headheight}{12.1638pt}

\titleformat{\paragraph}
{\normalfont\normalsize\bfseries}{\theparagraph}{1em}{}
\titlespacing*{\paragraph}
{0pt}{3.25ex plus 1ex minus .2ex}{1.5ex plus .2ex}

\newtheoremstyle{break}
  {\topsep}{\topsep}%
  {\itshape}{}%
  {\bfseries}{}%
  {\newline}{}%
\theoremstyle{break}

\definecolor{MyGray}{rgb}{0.87,0.87,0.87}

\declaretheorem[name=Définition,shaded={bgcolor=MyGray}]{defn}

\declaretheorem[name=Théorème,shaded={bgcolor=MyGray}]{thrm}

\declaretheorem[name=Proposition,shaded={bgcolor=MyGray}]{prop}

\declaretheorem[name=Exemple,numbered=no]{exmp}

\declaretheorem[name=Démonstration,numbered=no]{demo}

%\newtheorem{defn}{Définition}[section]
%\newtheorem*{exmp}{Exemple}
%\newtheorem{thrm}{Théorème}[section]
%\newtheorem*{demo}{Démonstration}
\newtheorem*{corll}{Corollaire}
\newtheorem*{exo}{Exercice}
%\newtheorem{prop}{Proposition}[section]
\newtheorem*{lemme}{Lemme}

\begin{document}
	\maketitle       % Titre
	\newpage         % Saut de page
	\tableofcontents % table des matières / Besoin d'une double compilation
	\newpage         % Saut de page
	
\section{Séance 1}

\begin{exo}
Construisez un graphe simple et connexe sur $8$ sommets tel que chaque sommet est contenu dans exactement trois ar\^etes. Pouvez-vous faire la m\^eme chose avec $9$ sommets?
\end{exo}

\begin{figure}[!h]
\centering
\scalebox{.825}{\begin{tikzpicture}

\tikzstyle{every node}=[circle, draw, fill=white, inner sep=0pt, minimum size=5pt]

\node (v2) at (0,0) {};
\node (v3) at (2,-1) {};
\node (v1) at (-2,-1) {};
\node (v8) at (-3,-3) {};
\node (v4) at (3,-3) {};
\node (v7) at (-2,-5) {};
\node (v5) at (2,-5) {};
\node (v6) at (0,-6) {};
\draw (v1) -- (v2) -- (v3) -- (v4) -- (v5) -- (v6) -- (v7) -- (v8) -- (v1);
\draw (v2) edge (v6);
\draw  (v5) edge (v1);
\draw  (v8) edge (v4);
\draw  (v7) edge (v3);
\end{tikzpicture}}
\end{figure}

Pas possible pour n=9.

%-----------------------------------------------------------------

\begin{exo}
Dans un groupe de personnes, il y a toujours deux individus qui connaissent exactement le m\^eme nombre de membres du groupe.
\begin{enumerate}
\item Formalisez cette propri\'et\'e dans le vocabulaire des graphes.
\item D\'emontrez cette propri\'et\'e (par l'absurde).
\end{enumerate}
\end{exo}

\begin{enumerate}
\item Soit $\Gamma$ un graphe à n sommets. $\exists u,v \in V(\Gamma)$ tq $deg(u)=deg(v)$
\item Par l'absurde: Supposons que $\cancel{\exists}{} e_{1}, e_{2}: deg(e_{1}) = deg(e_{2})$. \\

Les degrés sont tous compris entre 0 et n-1 (c'est à dire qu'on a un sommet pour chaque degré). Il existe donc un sommet qui est isolé (celui de degré 0) et un sommet qui est relié à ``tous'' les autres sommets (celui de degré n-1), ce qui est impossible.
\end{enumerate}

%-----------------------------------------------------------------

\begin{exo}
Soit $n\geq 2$ et soit $G$ un graphe simple avec $2n$ sommets et $n^2+1$ ar\^etes. Montrez que $G$ contient un triangle.
\end{exo}

%-----------------------------------------------------------------

\begin{exo}
Soit $G$ un graphe simple avec $2p$ sommets. On suppose que le degr\'e de chaque sommet est au moins \'egal \`a $p$. D\'emontrez que ce graphe est connexe.
\end{exo}

Démontrons par l'absurde: Supposons que le graphe ne soit pas connexe.\\

Soient x et y deux sommets tels qu'il n'existe pas de chemin entre x et y. Vu que $\forall v: deg(v)\geq p$, x a au moins p voisins (et y aussi). Les voisins de x sont différents des voisins de y, sinon il existerait un chemin entre x et y.\\

Le graphe est donc composé de $\underbrace{1+p}_{\text{x et ses voisins}}+\underbrace{1+p}_{\text{y et ses voisins}} = 2p+2$ sommets. Ceci est impossible, car l'énoncé dit que le graphe est composé de 2p sommets.

%-----------------------------------------------------------------

\begin{exo}
Soit $G$ un graphe simple.
\begin{enumerate}
\item On suppose que $G$ est connexe et que $x$ est un sommet de $G$ de degr\'e $1$. Prouvez que $G\setminus\{x\}$ est connexe.
\item D\'eduisez-en que, si $G$ est connexe et $|V(G)|=n\geq 2$, alors $G$ contient au moins $n-1$ ar\^etes.
\end{enumerate}
\end{exo}

%-----------------------------------------------------------------

\begin{exo}
Donnez un graphe simple et connexe sur au moins $5$ sommets qui est:
\begin{itemize}
\item hamiltonien et eul\'erien;
\item hamiltonien et non eul\'erien;
\item non hamiltonien et eul\'erien;
\item non hamiltonien et non eul\'erien.
\end{itemize}
\end{exo}

%-----------------------------------------------------------------

\begin{exo}
Le graphe 1 est-il isomorphe \`a un (ou \`a plusieurs) des graphes ci-dessous~?
\end{exo}

\begin{figure}[!h]
\centering
\scalebox{.825}{
\begin{tikzpicture}

\tikzstyle{every node}=[circle, draw, fill=white, inner sep=0pt, minimum size=5pt]

\node (a00) at (0,0) {};
\node (aleft) at (-1,0) {};
\node (aabove) at (0,1) {};
\node (adown) at (0,-1) {};
\node (aright) at (1,0) {};
\node (aright2) at (2,0) {};

\draw (a00) -- (aleft);
\draw (a00) -- (aright);
\draw (a00) -- (adown);
\draw (a00) -- (aabove);
\draw (aright) -- (aright2);
\end{tikzpicture}}
\caption{}
\end{figure}

\begin{figure}[!h]
\scalebox{.825}{
\begin{minipage}[t]{0.2\linewidth}
   \begin{tikzpicture}
		\tikzstyle{every node}=[circle, draw, fill=white, inner sep=0pt, minimum size=5pt]

        \node (a00) at (0,0) {};
        \node (leftup) at (-1,1) {};
        \node (leftdown) at (-1,-1) {};
        \node (a01) at (1,0) {};
        \node (rightup) at (2,1) {};
        \node (rightdown) at (2,-1) {};

		\draw (a00) -- (leftup);
		\draw (a00) -- (leftdown);
		\draw (a00) -- (a01);
		\draw (a01) -- (rightup);
		\draw (a01) -- (rightdown);
	\end{tikzpicture}
\end{minipage}

\begin{minipage}[t]{0.2\linewidth}
   \begin{tikzpicture}
		\tikzstyle{every node}=[circle, draw, fill=white, inner sep=0pt, minimum size=5pt]

        \node (a0) at (0,0) {};
        \node (a1) at (1,0) {};
        \node (a0side) at (-1,0) {};
        \node (a0up) at (0,1) {};
        \node (a0down) at (0,-1) {};
        \node (a1side) at (2,0) {};
        \node (a1up) at (1,1) {};
        \node (a1down) at (1,-1) {};

		\draw (a0side) -- (a0) -- (a1) -- (a1side);
		\draw (a0up) -- (a0) -- (a0down);
		\draw (a1up) -- (a1) -- (a1down);

	\end{tikzpicture}
\end{minipage}

\begin{minipage}[t]{0.2\linewidth}
   \begin{tikzpicture}
		\tikzstyle{every node}=[circle, draw, fill=white, inner sep=0pt, minimum size=5pt]

        \node (a00) at (0,0) {};
		\node (a0up) at (0,1) {};
		\node (a0down) at (0,-1) {};
		\node (a0ddown) at (0,-2) {};
		\node (a0left) at (-1,1) {};
		\node (a0right) at (1,1) {};

		\draw (a0up) -- (a00) -- (a0down) -- (a0ddown);
		\draw (a00) -- (a0left);
		\draw (a00) -- (a0right);
	\end{tikzpicture}
\end{minipage}

\begin{minipage}[t]{0.2\linewidth}
	\hspace{-1cm}
   \begin{tikzpicture}
		\tikzstyle{every node}=[circle, draw, fill=white, inner sep=0pt, minimum size=5pt]

        \node (a00) at (0,0) {};
        \node (a0up) at (0,1) {};
        \node (a0down) at (0,-1) {};
        \node (a0left) at (-1,0) {};
        \node (a0right) at (1,0) {};
        \node (a0rright) at (2,0) {};

		\draw (a0left) -- (a00) -- (a0right) -- (a0rright);
		\draw (a0up) -- (a00) -- (a0down);
		\draw (a0up) -- (a0left) -- (a0down);
	\end{tikzpicture}
\end{minipage}

\begin{minipage}[t]{0.2\linewidth}
   \begin{tikzpicture}
		\tikzstyle{every node}=[circle, draw, fill=white, inner sep=0pt, minimum size=5pt]

        \node (a00) at (0,0) {};
		\node (a0up) at (0,1) {};
		\node (a0down) at (0,-1) {};
		\node (a0ddown) at (0,-2) {};
		\node (a0left) at (-1,-1) {};
		\node (a0right) at (1,0) {};

		\draw (a0up) -- (a00) -- (a0down) -- (a0ddown);
		\draw (a00) -- (a0left);
		\draw (a0down) -- (a0right);
	\end{tikzpicture}
\end{minipage}

\begin{minipage}[t]{0.2\linewidth}
   \begin{tikzpicture}
		\tikzstyle{every node}=[circle, draw, fill=white, inner sep=0pt, minimum size=5pt]

        \node (a00) at (0,0) {};
		\node (a0left) at (-1,0) {};
		\node (a0right) at (1,0) {};
		\node (a0rright) at (2,0) {};
		\node (a0upleft) at (0,1) {};
		\node (a0upright) at (2,1) {};

		\draw (a0left) -- (a00) -- (a0right) -- (a0rright);
		\draw (a0upleft) -- (a0right) -- (a0upright);
	\end{tikzpicture}
\end{minipage}}
\caption{}
\end{figure}

Le troisième et le sixième.

\newpage

%-----------------------------------------------------------------

\begin{exo}
Les graphes suivants sont-ils isomorphes~? (Ne vous contentez pas d'une justification approximative: essayez de d\'emontrer rigoureusement vos affirmations.)
\end{exo}

\begin{figure}[!h]
\centering

\begin{minipage}[t]{0.2\linewidth}
	\begin{tikzpicture}[scale=0.4]
		\tikzstyle{every node}=[circle, draw, fill=white, inner sep=0pt, minimum size=5pt]

		\node (v6) at (-3.5,2.5) {};
		\node (v5) at (-5.5,2.5) {};
		\node (v8) at (-5.5,0.5) {};
		\node (v7) at (-3.5,0.5) {};
		\node (v1) at (-7,4) {};
		\node (v2) at (-2,4) {};
		\node (v4) at (-7,-1) {};
		\node (v3) at (-2,-1) {};
		\draw  (v1) edge (v2);
		\draw  (v3) edge (v2);
		\draw  (v4) edge (v3);
		\draw  (v1) edge (v4);
		\draw  (v5) edge (v6);
		\draw  (v6) edge (v7);
		\draw  (v7) edge (v8);
		\draw  (v8) edge (v5);
		\draw  (v5) edge (v1);
		\draw  (v6) edge (v2);
	\end{tikzpicture}
\end{minipage}
\hspace{1.5cm}
\begin{minipage}[t]{0.2\linewidth}
	\begin{tikzpicture}[scale=0.4]
		\tikzstyle{every node}=[circle, draw, fill=white, inner sep=0pt, minimum size=5pt]

		\node (v6) at (-3.5,2.5) {};
		\node (v5) at (-5.5,2.5) {};
		\node (v8) at (-5.5,0.5) {};
		\node (v7) at (-3.5,0.5) {};
		\node (v1) at (-7,4) {};
		\node (v2) at (-2,4) {};
		\node (v4) at (-7,-1) {};
		\node (v3) at (-2,-1) {};
		\draw  (v1) edge (v2);
		\draw  (v3) edge (v2);
		\draw  (v4) edge (v3);
		\draw  (v1) edge (v4);
		\draw  (v5) edge (v6);
		\draw  (v6) edge (v7);
		\draw  (v7) edge (v8);
		\draw  (v8) edge (v5);
		\draw  (v5) edge (v1);
		\draw  (v7) edge (v3);
	\end{tikzpicture}
\end{minipage}
\caption{}
\end{figure}%

Non isomorphes. \\

\begin{figure}[!h]
\centering

\begin{minipage}[t]{0.2\linewidth}
   \begin{tikzpicture}[scale=0.6]
		\tikzstyle{every node}=[circle, draw, fill=white, inner sep=0pt, minimum size=5pt]

        \node (v6) at (0,0) {};
		\node (v7) at (1.5,0) {};
		\node (v5) at (0,-1.5) {};
		\node (v1) at (-1,1) {};
		\node (v2) at (2.5,1) {};
		\node (v4) at (-1,-2.5) {};
		\node (v3) at (2.5,-2.5) {};
		\draw (v1) -- (v2) -- (v2) -- (v3) -- (v3) -- (v4) -- (v4) -- (-1,1);
		\draw (v5) -- (v6) -- (v6) -- (v7) -- (v7) -- (v3) -- (v3) -- (v5) -- (v6) -- (v1);
		\draw (v2) -- (v7);
	\end{tikzpicture}
\end{minipage}
\hspace{1.5cm}
\begin{minipage}[t]{0.2\linewidth}
   \begin{tikzpicture}[scale=0.6];
		\tikzstyle{every node}=[circle, draw, fill=white, inner sep=0pt, minimum size=5pt]

        \node (v1) at (-0.5,0.25) {};
		\node (v2) at (-1.5,2) {};
		\node (v5) at (-1.5,-1.5) {};
		\node (v6) at (1,1) {};
		\node (v3) at (2,2) {};
		\node (v7) at (1,-0.5) {};
		\node (v4) at (2,-1.5) {};
		\draw (v2) -- (v3) -- (v4) -- (v5) -- (v2) -- (v1) -- (v6) -- (v7) -- (v1) -- (v5);
		\draw (v3) -- (v6) -- (v7) -- (v4);

	\end{tikzpicture}
\end{minipage}
\caption{}
\end{figure}

Non isomorphes.\\

\begin{figure}[!h]
\centering
\begin{minipage}[t]{0.2\linewidth}
   \begin{tikzpicture}[scale=2]
   		\node[draw=none,minimum size=3cm,regular polygon,regular polygon sides=7] (a) {};

		\foreach \x in {1,2,...,7}
			\draw[fill=white,inner sep=0pt, minimum size=5pt] (a.corner \x) circle (1.5pt);
		  
		\draw (a.corner 1) -- (a.corner 2) -- (a.corner 3) -- (a.corner 4) -- (a.corner 5) -- (a.corner 6) -- (a.corner 7) -- (a.corner 1);
		\draw (a.corner 2) -- (a.corner 7) -- (a.corner 5) -- (a.corner 3) -- (a.corner 1) -- (a.corner 6) -- (a.corner 4) -- (a.corner 2) ;

	\end{tikzpicture}
\end{minipage}
\hspace{1.5cm}
\begin{minipage}[t]{0.2\linewidth}
   \begin{tikzpicture}[scale=2]
		\node[draw=none,minimum size=3cm,regular polygon,regular polygon sides=7] (a) {};

		\foreach \x in {1,2,...,7}
  			\draw[fill=white,inner sep=0pt, minimum size=5pt] (a.corner \x) circle (1.5pt);

		\draw (a.corner 1) -- (a.corner 5) -- (a.corner 2) -- (a.corner 6) -- (a.corner 3) -- (a.corner 7) -- (a.corner 4) -- (a.corner 1);
		\draw (a.corner 1) -- (a.corner 2) -- (a.corner 3) -- (a.corner 4) -- (a.corner 5) -- (a.corner 6) -- (a.corner 7) -- (a.corner 1);


	\end{tikzpicture}
\end{minipage}
\caption{}
\end{figure}

Isomorphes.\\ \newpage 
\documentclass[11pt, a4paper]{article} 

%\parindent 5mm 
%\parskip 0mm 
%\addtolength{\textwidth}{3 truecm} 
%\addtolength{\textheight}{1 truecm} 
%\setlength{\voffset}{-.6 truecm} 
%\setlength{\hoffset}{-1.3 truecm} 

\usepackage[lmargin=25mm,rmargin=25mm,tmargin=20mm,bmargin=20mm]{geometry}

\usepackage[frenchb]{babel} 
\usepackage{amsthm, amssymb, amsmath} 
\usepackage[T1]{fontenc} 
\usepackage{tikz, pgf} 
\usepackage{graphicx} 
\usepackage{xcolor} 
\usepackage{url, enumerate} 
\usepackage{epsfig}
\usepackage{tikz}

\begin{document} 

\title{MATH-F-307: S\'eance d'exercices 2} 
\author{} 
\date{2 octobre 2015}

\theoremstyle{plain} 
\newtheorem*{theo}{Th\'eor\`eme}
%\newtheorem{defi}[theo]{D\'efinition} 
%\newtheorem{lemme}[theo]{Lemme} 
%\newtheorem{corol}[theo]{Corollaire} 
%\newtheorem{prop}[theo]{Proposition} 
%\newtheorem{fait}[theo]{Fait} 
%\newtheorem{aff}{Affirmation} 
%\newtheorem*{rapp}{Rappel} 
%\newtheorem{conj}[theo]{Conjecture} 

\theoremstyle{definition} 
%\newtheorem{probl}[theo]{Probl\`eme ouvert} 
%\newtheorem*{ex}{Exemple} 
%\newtheorem*{rem}{Remarque} 
\newtheorem{exo}{Exercice}

\newcommand{\R}{\mathbb{R}} 
\newcommand{\Z}{\mathbb{Z}} 
\newcommand{\N}{\mathbb{N}} 

%\renewcommand{\theenumi}{(\alph{enumi})}
%\renewcommand{\labelenumi}{\theenumi}
\renewcommand{\FrenchLabelItem}{\textbullet}

\maketitle


\begin{exo}
Terminez les exercices de la s\'eance 1.
\end{exo}

\begin{exo}
Consid\'erez la grille $n \times n$, le graphe obtenu selon la Figure~\ref{fig:grille}, avec $n$ un naturel~$\geq 3$.
D\'emontrez que $n$ est pair si et seulement si le graphe est hamiltonien.


\begin{figure}[!h]
\centering
\begin{tikzpicture}[scale = 0.5]
\tikzstyle{every node}=[draw,circle,fill=black,minimum size=4pt,inner sep=0pt]
\node (a00) at (0,0) [shape= circle, draw, fill = white] {};
\node (a10) at (1,0) [shape= circle, draw, fill = white] {};
\node (a20) at (2,0) [shape= circle, draw, fill = white] {};
\node (a30) at (3,0) [shape= circle, draw, fill = white] {};
\node (a40) at (4,0) [shape= circle, draw, fill = white] {};

\node (a01) at (0,1) [shape= circle, draw, fill = white] {};
\node (a11) at (1,1) [shape= circle, draw, fill = white] {};
\node (a21) at (2,1) [shape= circle, draw, fill = white] {};
\node (a31) at (3,1) [shape= circle, draw, fill = white] {};
\node (a41) at (4,1) [shape= circle, draw, fill = white] {};

\node (a02) at (0,2) [shape= circle, draw, fill = white] {};
\node (a12) at (1,2) [shape= circle, draw, fill = white] {};
\node (a22) at (2,2) [shape= circle, draw, fill = white] {};
\node (a32) at (3,2) [shape= circle, draw, fill = white] {};
\node (a42) at (4,2) [shape= circle, draw, fill = white] {};

\node (a03) at (0,3) [shape= circle, draw, fill = white] {};
\node (a13) at (1,3) [shape= circle, draw, fill = white] {};
\node (a23) at (2,3) [shape= circle, draw, fill = white] {};
\node (a33) at (3,3) [shape= circle, draw, fill = white] {};
\node (a43) at (4,3) [shape= circle, draw, fill = white] {};

\node (a04) at (0,4) [shape= circle, draw, fill = white] {};
\node (a14) at (1,4) [shape= circle, draw, fill = white] {};
\node (a24) at (2,4) [shape= circle, draw, fill = white] {};
\node (a34) at (3,4) [shape= circle, draw, fill = white] {};
\node (a44) at (4,4) [shape= circle, draw, fill = white] {};

\draw (a00) -- (a01) -- (a02) -- (a03) -- (a04);
\draw (a10) -- (a11) -- (a12) -- (a13) -- (a14);
\draw (a20) -- (a21) -- (a22) -- (a23) -- (a24);
\draw (a30) -- (a31) -- (a32) -- (a33) -- (a34);
\draw (a40) -- (a41) -- (a42) -- (a43) -- (a44);

\draw (a00) -- (a10) -- (a20) -- (a30) -- (a40);
\draw (a01) -- (a11) -- (a21) -- (a31) -- (a41);
\draw (a02) -- (a12) -- (a22) -- (a32) -- (a42);
\draw (a03) -- (a13) -- (a23) -- (a33) -- (a43);
\draw (a04) -- (a14) -- (a24) -- (a34) -- (a44);
\end{tikzpicture}
\caption{Grille $5 \times 5$.}
\label{fig:grille}
\end{figure}
\end{exo}

\begin{exo}
Prouvez que pour tout $n \geq 3$, le graphe complet $K_n$ poss\`ede exactement $\frac{1}{2}(n-1)!$ cycles hamiltoniens.
\end{exo}

\begin{exo}
Combien d'arbres couvrants poss\`edent les deux graphes de la Figure~\ref{fig:arbrecouvrant}?
\begin{figure}[!h]
\centering
\begin{tikzpicture}[scale = 0.6]
\tikzstyle{every node}=[draw,circle,fill=black,minimum size=4pt,inner sep=0pt]
\node (a1) at (0,0) [shape= circle, draw, fill = white] {};
\node (a2) at (1,0) [shape= circle, draw, fill = white] {};
\node (a3) at (2,1) [shape= circle, draw, fill = white] {};
\node (a4) at (3,0) [shape= circle, draw, fill = white] {};
\node (a5) at (4,0) [shape= circle, draw, fill = white] {};
\node (a6) at (2,-1) [shape= circle, draw, fill = white] {};
\draw (a1) -- (a2) -- (a3) -- (a4) -- (a5);
\draw (a2) -- (a6) -- (a4);

\node (b1) at (7,-1) [shape= circle, draw, fill = white] {};
\node (b2) at (9,-1) [shape= circle, draw, fill = white] {};
%\node (b3) at (10,0) [shape= circle, draw, fill = white] {};
\node (b4) at (9,1) [shape= circle, draw, fill = white] {};
\node (b5) at (7,1) [shape= circle, draw, fill = white] {};
\draw (b1) -- (b2); -- (b3) -- 
\draw (b4) -- (b5) -- (b1) -- (b4) -- (b2) -- (b5);
\end{tikzpicture}
\caption{}
\label{fig:arbrecouvrant}
\end{figure}
\end{exo}

\begin{exo}
Montrez que tous les alcools $C_nH_{2n+1}OH$ sont des mol\'ecules dont le graphe est un arbre, en sachant que les valences de $C, O$ et de $H$ sont respectivement $4, 2, 1$.
\end{exo}

\begin{exo}
D\'emontrez que si un graphe hamiltonien $G = (V,E)$ est biparti selon la bipartition $V = A \cup B$, alors $|A|= |B|$. En d\'eduire que $K_{n,m}$, le graphe biparti complet, est hamiltonien si et seulement si $m=n \geq 2$.

\end{exo}

\begin{exo}
Pour chaque graphe de la Figure~\ref{fig:graphes}, d\'eterminez si 
\begin{enumerate}
\item le graphe est hamiltonien,
\item le graphe est eul\'erien,
\item le graphe est biparti.
\end{enumerate}
\begin{figure}[!h]
\centering
\begin{tikzpicture}[scale = 0.5]
\tikzstyle{every node}=[draw,circle,fill=black,minimum size=4pt,inner sep=0pt]
\node (a1) at (0,4) [shape= circle, draw, fill = white] {};
\node (a2) at (-2,0) [shape= circle, draw, fill = white] {};
\node (a3) at (2,0) [shape= circle, draw, fill = white] {};
\node (a4) at (0.6,1.8) [shape= circle, draw, fill = white] {};
\node (a5) at (-0.6,1.8) [shape= circle, draw, fill = white] {};
\node (a6) at (0,1.4) [shape= circle, draw, fill = white] {};
\draw (a1) -- (a2) -- (a3) -- (a1) -- (a5) -- (a2);
\draw (a2) -- (a6) -- (a4);
\draw (a3) -- (a4) -- (a5) -- (a6);

\node (a1) at (5,0) [shape= circle, draw, fill = white] {};
\node (a2) at (7,0) [shape= circle, draw, fill = white] {};
\node (a3) at (9,0) [shape= circle, draw, fill = white] {};
\node (a4) at (11,0) [shape= circle, draw, fill = white] {};
\node (b1) at (5,4) [shape= circle, draw, fill = white] {};
\node (b2) at (7,4) [shape= circle, draw, fill = white] {};
\node (b3) at (9,4) [shape= circle, draw, fill = white] {};
\node (b4) at (11,4) [shape= circle, draw, fill = white] {};
\draw (a1) -- (a2) -- (a3) -- (a4) -- (b4) -- (b3) -- (b2) -- (b1) -- (a1) -- (b2);
\draw (a1) -- (b3) -- (a4) -- (b1) -- (a3) -- (b3) ;
\draw (b2) -- (a2) -- (b4);


\node (a1) at (14,-0.5) [shape= circle, draw, fill = white] {};
\node (a2) at (14,4.5) [shape= circle, draw, fill = white] {};
\node (a3) at (19,4.5) [shape= circle, draw, fill = white] {};
\node (a4) at (19,-0.5) [shape= circle, draw, fill = white] {};

\node (b1) at (15,0.5) [shape= circle, draw, fill = white] {};
\node (b11) at (16,0.5) [shape= circle, draw, fill = white] {};
\node (b2) at (15,3.5) [shape= circle, draw, fill = white] {};
\node (b33) at (18,2.5) [shape= circle, draw, fill = white] {};
\node (b3) at (18,3.5) [shape= circle, draw, fill = white] {};
\node (b4) at (18,0.5) [shape= circle, draw, fill = white] {};

\node (c1) at (16,1.5) [shape= circle, draw, fill = white] {};
\node (c2) at (16,2.5) [shape= circle, draw, fill = white] {};
\node (c3) at (17,2.5) [shape= circle, draw, fill = white] {};
\node (c4) at (17,1.5) [shape= circle, draw, fill = white] {};

\draw (a1) -- (a2) -- (a3) -- (a4) -- (a1);
\draw (a1) -- (b1) -- (b2) -- (b3) -- (a3);
\draw (a1) -- (b11) -- (b4) -- (b33) -- (a3);
\draw (a2) -- (b2) -- (c2);
\draw (a4) -- (b4) -- (c4);
\draw (c1) -- (c2) -- (c3) -- (c4) -- (c1);
\draw (b1) -- (c1) -- (b11);
\draw (b33) -- (c3) -- (b3);
\end{tikzpicture}
\caption{}
\label{fig:graphes}
\end{figure}

\end{exo}
\thispagestyle{empty}
\end{document} \newpage

\section{Séance 3}

\vspace*{1cm}

\begin{exo}
Construisez un code de Gray d'ordre $5$ sur base du code de Gray d'ordre $4$ ci-dessous.\\
0000,0100,1100,1000,1010,1110,0110,0010,0011,0111,1111,1011,1001,1101,0101,0001
\end{exo}

\vspace*{1cm}

%-----------------------------------------------------------------

\begin{exo}
Dans le graphe ci-dessous, on donne un couplage de cardinal maximal. En utilisant la preuve du th\'eor\`eme de K\"onig vue au cours, trouvez un transversal de cardinal minimal.
\end{exo}

\begin{figure}[!h]
\begin{center}
\begin{tikzpicture}[scale=0.8]

\tikzstyle{every node}=[draw,circle,minimum size=10pt,inner sep=2pt]
\tikzstyle{every label}=[position=above]

\node[circle] (v7) at (0,0) {D};
\node[circle] (v8) at (0,-1) {H};
\node[circle] (v10) at (0,-2) {F};
\node[circle] (v12) at (0,-3) {G};
\node[circle] (v5) at (0,1) {C};
\node[circle] (v1) at (0,2) {A};
\node[circle] (v3) at (3,2) {B};
\node[circle] (v4) at (3,0) {E};
\node[circle] (v9) at (3,-2) {I};
\node[circle] (v11) at (3,-3) {K};
\node[circle] (v6) at (3,-4) {L};
\node[circle] (v13) at (0,-4) {J};
\draw  (v1) -- (v3);
\draw  (v5) -- (v3);
\draw  (v8) -- (v9);
\draw  (v10) -- (v4);
\draw  (v12) -- (v4);

\draw[red] (v7) -- (v3);
\draw[red] (v5) -- (v4);
\draw[red] (v10) -- (v9);
\draw[red] (v12) -- (v6);
\draw[red] (v11) -- (v13);
\end{tikzpicture}
\end{center}
\caption{}
\end{figure}

%-----------------------------------------------------------------

\begin{exo}
Sur $\R^2$, on d\'efinit les relations suivantes:
$$(x,y)\mathcal{R}(x',y') \Leftrightarrow x \leq x' \mathrm{~et~} y \leq y', $$
$$(x,y)\mathcal{S}(x',y') \Leftrightarrow (x < x') \mathrm{~ou~} (x = x' \mathrm{~et~} y \leq y').$$
Est-ce que les relations $\mathcal{R}$ et $\mathcal{S}$ sont des ordres?
\end{exo}

\newpage

%-----------------------------------------------------------------

\begin{exo}
Consid\'erons le graphe biparti (bipartition donn\'ee par une coloration des sommets) ci-dessous. Sur l'ensemble de ses sommets, on d\'efinit la relation $u\leq v$ pour $u,v$ des sommets tels que $u$ est un sommet rouge et $\{u,v\}$ est une ar\^ete. On pose aussi $u\leq u$ pour tout sommet $u$. \\
\begin{enumerate}[(a)]
\item V\'erifiez que $\leq$ est un ordre partiel.
\item Construisez une partition des sommets par $k$ cha\^ines et trouvez une anticha\^ine contenant $k$ \'el\'ements.
\item D\'eduisez-en un couplage de cardinalit\'e maximale et un transversal de cardinalit\'e minimale. 
\item (Bonus) Sur base de ce qui est fait ci-dessus, prouvez que le th\'eor\`eme de K\"onig implique le th\'eor\`eme de Dilworth.
\end{enumerate}
\end{exo}

\begin{figure}[!h]
\begin{center}
\begin{tikzpicture}

\tikzstyle{every node}=[draw,circle,fill=black,minimum size=10pt,inner sep=2pt]
\tikzstyle{every label}=[position=above]

\node[circle,fill=red] (v7) at (0,0) {C};
\node[circle,fill=red] (v8) at (0,-1) {D};
\node[circle,fill=red] (v10) at (0,-2) {E};
\node[circle,fill=red] (v12) at (0,-3) {F};
\node[circle,fill=red] (v5) at (0,1) {B};
\node[circle,fill=red] (v1) at (0,2) {A};
\node[circle,fill=blue] (v3) at (3,1.5) {G};
\node[circle,fill=blue] (v2) at (3,2.5) {M};
\node[circle,fill=blue] (v4) at (3,0.5) {H};
\node[circle,fill=blue] (v9) at (3,-0.5) {I};
\node[circle,fill=blue] (v11) at (3,-1.5) {J};
\node[circle,fill=blue] (v6) at (3,-2.5) {K};
\node[circle,fill=blue] (v13) at (3,-3.5) {L};
\draw  (v1) edge (v2);
\draw  (v1) edge (v3);
\draw  (v1) edge (v4);
\draw  (v5) edge (v3);
\draw  (v5) edge (v6);
\draw  (v7) edge (v4);
\draw  (v8) edge (v9);
\draw  (v10) edge (v4);
\draw  (v10) edge (v11);
\draw  (v12) edge (v9);
\draw  (v12) edge (v13);
\end{tikzpicture}
\end{center}
\caption{}
\end{figure}

%----------------------------------------------------------------- \newpage

\section{Séance 4}

\begin{exo}
L'ensemble $\{2^m | m \in \Z\}$ forme-t-il un groupe lorsqu'il est muni de la multiplication usuelle?
\end{exo}

Soit $X=2^{m}, m \in \Z$ et $(X,.)$ le groupe à analyser. \\

$\forall x,y \in \Z : 2^{x} * 2^{y} = 2^{x+y} \in X$ car $(x+y) \in \Z$. L'ensemble X forme donc bien un groupe lorsqu'il est muni de la multiplication. 

%------------------------------------------------

\vspace*{0.8cm}
\begin{exo}
L'ensemble $(\Z/2\Z)^n=\{(x_1,x_2,\ldots,x_n)|x_1,x_2,\ldots,x_n\in \Z/2\Z\}$ avec l'addition d\'efinie par $$(x_1,x_2,\ldots,x_n)+(y_1,y_2,\ldots,y_n)=(x_1+y_1,x_2+y_2,\ldots,x_n+y_n)$$
(o\`u $x_i+y_i$ est le r\'esultat d'une addition modulo $2$) forme-t-il un groupe?
\end{exo}

Il faut tester si les 3 propriétés d'un groupe sont respectées.

\begin{enumerate}
\item Associativité: Chaque composante est calculée avec la forme $x_{i}+y_{i}, \forall i \in \{ 1,2,...,n \}$. $\Z_{2}$ est associatif, l'adition est faite composante par composante, donc $Z_{2}^{n}$ est associatif. Il faut donc à présent montrer que $(x_{i}+y_{i}) + z_{i} = x_{i} + (y_{i} + z_{i})$.
\end{enumerate}

%------------------------------------------------

\vspace*{0.8cm}
\begin{exo}
En appliquant l'algorithme d'Euclide \`a $a$ et $b$ ci-dessous, calculer :
\begin{itemize}
\item[-] le PGCD($a,b$),
\item[-] $x$ et $y$ tels que $ ax + by = \text{PGCD}(a,b),$
\end{itemize}
Les diff\'erentes valeurs de $a$ et $b$ sont :
\begin{enumerate}[(i)]
\item $a = 12, b = 34$,
\item $a = 13, b = 34$,
\item $a = 13, b = 31$,
\end{enumerate}
\end{exo}

\vspace*{0.8cm}
\begin{exo} 
\begin{enumerate}[(i)]
\item Trouver un entier $x$ tel que le reste de la division de $50x$ par $71$ donne $1$.
\item Trouver un entier $x$ tel que le reste de la division de $50x$ par $71$ donne $63$.
\item Trouver un entier $x$ tel que le reste de la division de $43x$ par $64$ donne $1$.
\end{enumerate}
\end{exo}

\vspace*{0.8cm}
\begin{exo}
Dans le syst\`eme RSA, prenons $p=11$, $q=13$ et $e=7$. Que vaut alors $s$? Si $99$ est le message \`a coder, quel est le message crypt\'e? V\'erifier en d\'ecriptant le message.
\end{exo}

\thispagestyle{empty} \newpage

\section{Séance 5}

\begin{exo}
Montrer le r\'esultat suivant: si $a\equiv b ~(mod~n)$ et $c\equiv d ~(mod~n)$, alors
$$ a+c \equiv b+d ~(mod~n) ~\mathrm{et}~ a.c \equiv b.d ~(mod~n).$$
\end{exo}

\vspace*{0.8cm}
\begin{exo}
Montrer que, si $a\equiv b ~(mod~n)$, alors $$a+c\equiv b+c ~(mod~n) ~\forall c \in \Z$$ et $$a.c\equiv b.c ~(mod~n) ~\forall c \in \Z.$$ 
\end{exo}

\vspace*{0.8cm}
\begin{exo}
Prouver que, si $a\equiv b ~(mod~n)$, alors $a^k\equiv b^k ~(mod~n)$ pour tout entier $k>0$.
\end{exo}

\vspace*{0.8cm}
\begin{exo}
Trouver toutes les solutions aux congruences suivantes:
\begin{itemize}
\item $2x \equiv 3 ~(mod~4)$ avec $x \in \Z/4\Z$;
\item $2x \equiv 2 ~(mod~4)$ avec $x \in \Z/4\Z$;
\item $2x \equiv 3 ~(mod~5)$ avec $x \in \Z/5\Z$.
\end{itemize}
Que pouvez-vous en d\'eduire?
\end{exo}

\vspace*{0.8cm}
\begin{exo}
Soient $a,b$ deux entiers. Montrer que $$a\Z \cap b\Z = ppcm(a,b)\Z$$ et $$ a\Z + b\Z = pgcd(a,b)\Z.$$
\end{exo}

\vspace*{0.8cm}
\begin{exo}
Montrer que $$\Z/3\Z \times \Z/3\Z \ncong \Z/9\Z$$ mais que $$\Z/2\Z \times \Z/3\Z \cong \Z/6\Z.$$

\end{exo} \newpage
\section{Séance 6 et 7}

\begin{exo}
Soient $A$ et $B$ deux ensembles finis avec $|A|=a$ et $|B|=b$ ($a,b \in \mathbb{N}$). Que valent:
\begin{enumerate}[(i)]
\item $|A\times B|$,
\item $|B^A|$ o\`u $B^A:=\{f:A \rightarrow B\}$,
\item $|\{f:A \rightarrow B: f \mathrm{~est~une~injection~de~} A \mathrm{~dans~}B\}|$,
\item $|\mathrm{Sym}\,A|$ o\`u $\mathrm{Sym}\,A$ est l'ensemble des permutations de $A$.
\end{enumerate}
\end{exo}

\begin{enumerate}[(i)]
	\item $a\cdot b$
	\item $b^a$
	\item Si $\#B < \#A$ pas d'injection possible, donc vaut zéro. Sinon, $(b-a+1)\cdot ... \cdot (b-a) \cdot b$ = $ \frac{b!}{(b-a)!}$
	\item $a!$
\end{enumerate}

%----------------------------------------------

\begin{exo}
Quels sont les ensembles $F$ non vides ayant la propri\'et\'e suivante:
\begin{enumerate}[(i)]
\item pour tout ensemble $X$, $|F^X|=1$?
\item pour tout ensemble $Y$, $|Y^F|=1$?
\end{enumerate}
\end{exo}

\begin{enumerate}[(i)]
	\item $|F| = 1 $
	\item Ensemble vide (mais pas possible par énoncé)
\end{enumerate}

%----------------------------------------------

\begin{exo}
Soient $f:A \rightarrow B$ et $g:B \rightarrow C$ deux fonctions. D\'emontrer:
\begin{enumerate}[(i)]
\item $g \circ f$ injective $\Rightarrow$ $f$ injective;
\item $g \circ f$ surjective $\Rightarrow$ $g$ surjective;
\item $g \circ f$ bijective $\Rightarrow$ ($f$ injective et $g$ surjective).
\end{enumerate}
\end{exo}

\begin{enumerate}[(i)]
	\item Si f non injective, deux éléments $a_{1}$ et $a_{2}$ différents de A vont être envoyés par f sur un élément b de B. De plus, ces deux éléments vont être envoyés par g o f sur un même élément c de C, car g ( f ($a_1$)) = g( b ) = c = g( b) = g ( f ( $a_2$))
	\item On sait que $\forall c \in C , \exists a \in A$ tel que g o f (a) = c. 
		On veut montrer que g est surjective. C'est à dire que $\forall c \in C, \exists b \in B$ tel que g (b) = c. 
		Ceci est vérifié en prenant b = f(a).
	\item Implication de (i) et (ii)
\end{enumerate}

%----------------------------------------------

\begin{exo}
Donner une preuve bijective de l'identit\'e de somme parall\`ele ${k \choose k} + {k+1 \choose k} + \cdots + {m \choose k} = {m+1 \choose k+1}$.
\end{exo}

Voir syllabus année passée page 8.

%----------------------------------------------

\begin{exo}
Donner deux d\'emonstrations de
$$
\sum_{k=0}^n {n \choose k} = 2^n\ .
$$
\end{exo}

Première démonstration: \\

Via le Binôme de Newton, on sait que\\

$$(x+y)^n = \sum_{k=0}^n {n \choose k} x^{n-k} y^k$$

Si on pose x=1 et y=1, on a: \\

\begin{align*}
 (1+1)^n &= \sum_{k=0}^n {n \choose k} \cdot \underbrace{1^{n-k}}_{=1} \cdot \underbrace{1^k}_{=1} \\
 2^n &= \sum_{k=0}^n {n \choose k}
\end{align*}

Deuxième démonstration: \\

${n \choose k}$ est le nombre de sous-ensembles à k éléments d'un ensemble à n éléments. \\

$$ |\{ 0,1 \}^n| = 2^n $$

%----------------------------------------------

\begin{exo}
Qu'obtient-on comme identit\'e sur les coefficients binomiaux en \'ecrivant
$$
(x+y)^{2n} = (x+y)^n(x+y)^n\ ?
$$
\end{exo}

(Voir avec assistants)

%----------------------------------------------

\begin{exo}
Qu'obtient-on en d\'erivant la formule du bin\^ome~?
\end{exo}

(Voir avec assistants)

%----------------------------------------------

\begin{exo}
Trouver le nombre de solutions de l'\'equation $x + y + z + w = 15$, dans les naturels.
\end{exo}

$ { s + d - 1 \choose d - 1 } = { 15 + 4 - 1 \choose 4 - 1} = {18 \choose 3 }$

%----------------------------------------------

\begin{exo} 
Combien l'\'equation
$$
x + y + z + t + u = 60
$$
poss\`ede-t-elle de solutions enti\`eres $(x,y,z,t,u)$ telles que
$$
x > 0\ ,\quad y \geqslant 9\ , \quad z > -2\ , \quad t \geqslant 0 \quad \textrm{ et }  \quad u > 10 \quad ?
$$
\end{exo}

On doit procéder à un changement de variables.\\

${x}' = x-1 \Leftrightarrow x= {x}'+1 \qquad {y}' = y-9 \Leftrightarrow y= {y}'+9 \qquad {z}' = z+1 \Leftrightarrow z= {z}'-1 $

$\qquad \qquad \qquad \qquad \qquad {t}' = t \qquad {u}' = u-11 \Leftrightarrow u= {u}'+11$\\

${x}'+{y}'+{z}'+{t}'+{u}' = 60 - 1 - 9 + 1 - 11 = 40$\\

${s + d - 1 \choose d - 1} = {40 + 5 - 1 \choose 5 - 1} = {44 \choose 4}$
%----------------------------------------------

\begin{exo} 
Trouver le nombre de solutions de l'in\'equation
$$
x + y + z + t \leqslant 6
$$
%
\begin{enumerate}[(i)]
\item dans les naturels;
\item dans les entiers $>0$;
\item dans les entiers, avec comme contraintes suppl\'ementaires $x > 2$, $y > -2$, $z > 0$ et $t > -3$.
\end{enumerate}
\end{exo}

Même chose que les exos précédents (réponse dans un prochain épisode...).

%----------------------------------------------

\begin{exo} 
Avec les lettres du mot MISSISSIPPI, combien peut-on \'ecrire de mots diff\'erents de 11 lettres~?
\end{exo}

Lettres du mot: 1 M, 4 I, 4 S, 2 P \\

Mots de 11 lettres: $ \frac{11!}{(4!)(4!)(2!)(1!)}$

%----------------------------------------------

\begin{exo} 
Avec les lettres du mot 
%
\begin{center}
H\,U\,M\,U\,H\,U\,M\,U\,N\,U\,K\,U\,N\,U\,K\,U\,A\,P\,U\,A\,A
\end{center}
(``poisson'' en hawa\"\i{}en), combien peut-on \'ecrire de mots diff\'erents de 21 lettres ne comprenant pas deux lettres U c\^ote \`a c\^ote~?
\end{exo}

Faire mots de 12 lettres sans U. Rajouter probabilité de mettre les U dans les 13 places qui restent pour faire des mots de 21 lettres. Donc, ${13 \choose 9}$

%----------------------------------------------

\begin{exo} 
Si $0 \leqslant m \leqslant n$, que vaut
$$
\sum_{k=m}^n {k \choose m}{n \choose k}\quad ?
$$
(Hint~: essayer une preuve bijective.)
\end{exo}

Preuve version "étudiant": \\

\begin{align*}
 \sum_{k=m}^n {k \choose m}{n \choose k} &= \sum_{k=m}^n \frac{\cancel{k!}}{m!(k-m)!} \frac{n!}{\cancel{k!}(n-k)!} = \sum_{k=m}^n \frac{n!}{m!(k-m)!(n-k)!} \\
 &= \sum_{k=m}^n \frac{n!}{m!(k-m)!(n-k)!} \frac{(n-m)!}{(n-m)!} = {n \choose m} \sum_{k=m}^n \frac{(n-m)!}{(k-m)!(n-k)!} \\
 &= {n \choose m} \sum_{k=m}^n {n-m \choose k-m} \vspace{1cm} \qquad \text{On pose} \quad r=k-m \\
 &= {n \choose m} \sum_{r=0}^{n-m} {n-m \choose r} 1^{r} 1^{n-m-r} \qquad \text{Formule du binôme de Newton} \\
 &= {n \choose m} (1+1)^{n-m} = {n \choose m} 2^{n-m} \\
\end{align*}

Preuve bijective version assistants: \\

\begin{minipage}{0.4\textwidth}
\begin{tikzpicture}[scale=0.5]

\begin{scope}[shift={(3cm,-5cm)}, fill opacity=0.5,
  mytext/.style={text opacity=1,font=\large\bfseries}]

\draw[fill=green, draw = black] (0,0) circle (5);
\draw[fill=yellow, draw = black] (-1.5,0) circle (3);
\draw[fill=blue, draw = black] (-1,0) circle (1.5);

\node[mytext] at (0,4) (N) {N};
\node[mytext] at (-3.8,0) (K) {K};
\node[mytext] at (-1,0) (M) {M};
\end{scope}
\end{tikzpicture}
\end{minipage}
\begin{minipage}{0.6\textwidth}
Fixons $0 \leq m \leq n$\\
Comptons de 2 manières différentes le nombre de triples (M, K, N) où $M \subseteq K \subseteq N$ et |M| = m, |N| = n, |K| = k\\

\begin{enumerate}
\item On choisit un ensemble de taille m dans N: il y a ${n \choose m}$ façons de choisir
\item K peut avoir m éléments, m+1, ..., n éléments \\
\end{enumerate}
\end{minipage}

\vspace{1cm}

\textbf{1. On choisit un ensemble de taille m dans N: il y a ${n \choose m}$ façons de choisir}\\

Ensuite, nous complétons cet ensemble M pour obtenir K, c'est à dire il y a $$\sum_{k=0}^{n-m} {n-m \choose k} = 2^{n-m} \quad \text{choix}$$ \\

Donc au total il y a ${n \choose m} 2^{n-m}$ choix.\\

\textbf{2. K peut avoir m éléments, m+1, ..., n éléments} \\

\begin{enumerate}
\item \textbf{S'il y en a m:} on choisit m éléments parmi n et m éléments parmi ces m éléments, c'est à dire ${m \choose m}{n \choose m}$
\item \textbf{S'il y en a m+1:} on choisit m+1 éléments parmi n et m éléments parmi ces m+1 éléments, c'est à dire ${m+1 \choose m}{n \choose m+1}$
\item \textbf{S'il y en a n:} on choisit n éléments parmi n et m éléments parmi ces n éléments, c'est à dire ${n \choose m}{n \choose n}$ \\
\end{enumerate}

Il suffit de tout sommer (car "ou exclusif"). Donc: $$ \sum_{k=m}^n {k \choose m}{n \choose k} = {n \choose m} 2^{(n-m)}$$

\vspace{1cm}

%----------------------------------------------

\begin{exo} 
Si on jette simultan\'ement $n$ d\`es identiques, combien de r\'esultats diff\'erents peut-on obtenir~? (Deux r\'esultats sont consid\'er\'es comme \'equivalents s'ils ont le m\^eme nombre de 1, le m\^eme nombre de 2, \ldots, le m\^eme nombre de 6.)
\end{exo}

(Voir avec assistants) \newpage
\section{Séance 8 et 9}

\begin{exo}
De combien de fa\c{c}ons diff\'erentes peut-on monter un escalier de 30 marches, si on monte \`a chaque pas soit d'une seule marche soit de deux marches \`a la fois~?
\end{exo}

\begin{exo} 
Que vaut le d\'eterminant de la matrice $n \times n$ 
$$
\left( 
\begin{array}{rrrrrrrr}
1 &-1 &0  &0  &\cdots &0 &0 &0\\
1 & 1 &-1 &0  &\cdots &0 &0 &0\\
0 & 1 &1  &-1 &\cdots &0 &0 &0\\
\vdots & \vdots & \vdots & \vdots && \vdots & \vdots &\vdots\\
0 & 0 & 0 & 0 & \cdots & ~1 &~1 &-1\\
0 & 0 & 0 & 0 & \cdots & 0 &1 &1\\
\end{array}
\right) \qquad ?
$$
\end{exo}

\begin{exo} 
Que vaut
$$
\lim_{n \to \infty} \frac{F_{n+1}}{F_n} \quad ?
$$
\end{exo}

\begin{exo} 
Prouver que, pour tout entier $n \geqslant 1$, 
$$
\varphi^n = F_n \cdot \varphi + F_{n-1}\ ,
$$
o\`u $\varphi := \frac{1+\sqrt{5}}{2}$ est le {\DEF nombre d'or}.
\end{exo}

\begin{exo} 
Prouver que, pour tout entier $n \geqslant 3$,
$$
F_n > \varphi^{n-2}
$$
\end{exo}

\begin{exo}
R\'esoudre les r\'ecurrences 
%
\begin{enumerate}[(i)]
\item $a_n = \frac{1}{2} a_{n-1} + 1$ pour $n \geqslant 1$,\hfill
 $a_0 = 1$

\item $a_n = 5a_{n-1} - 6a_{n-2}$ pour $n \geqslant 2$,\hfill
 $a_0 = -1$, \quad $a_1 = 1$

\item $a_n = 6a_{n-1} - 9a_{n-2}$ pour $n \geqslant 2$,\hfill
 $a_0 = 1$, \quad $a_1 = 9$

\item $a_n = 4a_{n-1} - 3a_{n-2} + 2^n$ pour $n \geqslant 2$,\hfill
$a_0 = 1$, \quad $a_1 = 11$
\end{enumerate}
\end{exo}

\begin{exo}
R\'esoudre les r\'ecurrences 
%
\begin{enumerate}[(i)]
\item $a_{n+2} = 3 a_{n+1} + 4 a_{n}$ pour $n \geqslant 0$,\hfill
      $a_0 = 1$, \quad $a_1 = 3$

\item $a_{n+3} - 6 a_{n+2} + 11 a_{n+1} - 6 a_{n}=0$ pour $n \geqslant 0$,\hfill
      $a_0 = 2$, \quad $a_1 = 0$, \quad $a_2 = -2$

\item $a_{n+3} = 3 a_{n+1} - 2 a_{n}$ pour $n \geqslant 0$,\hfill
      $a_0 = 1$, \quad $a_1 = 0$, \quad $a_2 = 0$
      
\item $a_{n+3} + 3 a_{n+2} + 3 a_{n+1} + a_n = 0$

\item $a_{n+4} + 4 a_{n} = 0$     
\end{enumerate}
\end{exo}

\begin{exo}
R\'esoudre la r\'ecurrence
$$
\begin{array}{ll}
a_{n+2} - (2 \cos \alpha) a_{n+1} + a_n = 0\quad \forall n \geqslant 0\\
a_1 = \cos \alpha, \quad a_2 = \cos 2 \alpha
\end{array}
$$
\end{exo}

\begin{exo}
R\'esoudre les r\'ecurrences
%
\begin{enumerate}[(i)]
\item $a_n + 2 a_{n-1} = n+3$ pour $n \geqslant 1$\hfill
      $a_0 = 3$
      
\item $a_{n+2} + 8 a_{n+1} - 9 a_{n} = 8 \cdot 3^{n+1}$ pour $n \geqslant 0$\hfill
      $a_0 = 2$, \quad $a_1 = -6$
      
\item $a_{n+2} - 6 a_{n+1} + 9 a_{n} = 2^n + n$ pour $n \geqslant 0$

\item $n a_n = (n+3) a_{n-1} + n^2 + n$ pour $n \geqslant 1$            
\end{enumerate}
\end{exo}

\begin{exo}
Que vaut le d\'eterminant de la matrice $n \times n$
$$
\left( 
\begin{array}{rrrrrrrr}
2 &1 &0  &0  &\cdots &0 &0 &0\\
1 & 2 &1 &0  &\cdots &0 &0 &0\\
0 & 1 &2  &1 &\cdots &0 &0 &0\\
\vdots & \vdots & \vdots & \vdots && \vdots & \vdots &\vdots\\
0 & 0 & 0 & 0 & \cdots & 1 &2 &1\\
0 & 0 & 0 & 0 & \cdots & 0 &1 &2\\
\end{array}
\right) \qquad ?
$$
\end{exo}

\begin{exo}
Avec l'alphabet $\{A,B,C\}$, combien peut-on \'ecrire de mots de $n$ lettres dans lesquels on ne trouve pas
%
\begin{enumerate}[(i)]
\item deux lettres $A$ c\^ote-\`a-c\^ote~?
\item deux lettres $A$ ni deux lettres $B$ c\^ote-\`a-c\^ote~?
\item deux lettres $A$ ni deux lettres $B$ ni deux lettres $C$ c\^ote-\`a-c\^ote~?
\end{enumerate}
\end{exo}

\begin{exo}
Donner le comportement asymptotique des suites $T(n)$ pour chacune des r\'ecurrences suivantes~:
%
\begin{enumerate}[(i)]
\item $T(n) = 2T(\lceil n/2 \rceil) + n^2$
\item $T(n) = T(\lfloor 9n/10 \rfloor) + n$
\item $T(n) = 16T(\lceil n/4 \rceil) + n^2$
\item $T(n) = 7 T(\lceil n/3 \rceil) + n^2$
\item $T(n) = 7 T(\lceil n/2 \rceil) + n^2$
\item $T(n) = 2 T(\lfloor n/4 \rfloor) + \sqrt{n}$
\item $T(n) = T(n-1) + n$
\item $T(n) = T(\lfloor \sqrt{n} \rfloor) + 1$
\end{enumerate}
\end{exo}

\begin{exo}
R\'esoudre la r\'ecurrence
$$
\begin{array}{l}
a_n = \sqrt{a_{n-1} a_{n-2}} \quad \forall n \geqslant 2\\
a_0 = 1, \quad a_1 = 2
\end{array}
$$
\end{exo}

\begin{exo} (Examen ao\^ut 2011.)
Combien y a-t-il de matrices $2 \times n$ \`a coefficients entiers v\'erifiant les deux conditions suivantes~?
%
\begin{itemize}
\item Dans chacune des deux lignes, chacun des entiers $1$, $2$, \ldots, $n$ appara\^\i{}t une et une seule fois.
\item Dans chacune des $n$ colonnes, les deux coefficients diff\`erent d'au plus $1$.
\end{itemize}
\end{exo}

\begin{exo} (Examen ao\^ut 2011.)
Soient $x$ et $y$ deux naturels de $2n$ bits, c'est-\`a-dire dont l'\'ecriture binaire occupe au plus $2n$ bits. %Supposons que $x = \sum_{i=0}^{2n-1} x_i 2^i$ et $y = \sum_{i=0}^{2n-1} y_i 2^i$ sont les \'ecritures binaires de $x$ et $y$. Cela peut s'\'ecrire aussi $x = (x_{2n-1}x_{2n-2}\cdots{}x_1x_0)_2$ et $y =  (y_{2n-1}y_{2n-2}\cdots{}y_1y_0)_2$. Posons maintenant $X_0 := (x_{n-1}x_{n-2}\cdots{}x_1x_0)_2$, $X_1 := (x_{2n-1}x_{2n-2}\cdots{}x_{n+1}x_n)_2$, $Y_0 := (y_{n-1}y_{n-2}\cdots{}y_1y_0)_2$ et $Y_1 := (y_{2n-1}y_{2n-2}\cdots{}y_{n+1}y_n)_2$. Donc $X_0$, $X_1$, $Y_0$ et $Y_1$ sont quatre naturels de $n$ bits tels que $x = 2^n X_1 + X_0$ et $y = 2^n Y_1 + Y_0$.\medskip
Soient $X_0$, $X_1$, $Y_0$ et $Y_1$ quatre naturels de $n$ bits tels que $x = 2^n X_1 + X_0$ et $y = 2^n Y_1 + Y_0$.\medskip

\begin{enumerate}[a)]
\item V\'erifier que
%
$$
xy = (2^{2n} + 2^n) X_1 Y_1 + 2^n (X_1 - X_0)(Y_0 - Y_1) + (2^n + 1)X_0Y_0\ .
$$

\item Consid\'erons l'algorithme r\'ecursif qui multiplie les naturels $x$ et $y$ en appliquant l'\'equation ci-dessus. Soit $f(n)$ le nombre d'op\'erations \underline{simples} (additions ou soustractions de bits, d\'ecalages, comparaisons\ldots{} etc) n\'ecessaires pour le calcul r\'ecursif du produit $xy$ par le biais de cette \'equation. Ecrire une relation de r\'ecurrence du type ``diviser pour r\'egner'' pour $f(n)$. (Il n'est pas n\'ecessaire de calculer avec pr\'ecision le nombre d'op\'erations simples requises, calculer ce nombre \`a une constante pr\`es est suffisant.)

\item Sur base de cette r\'ecurrence, d\'eterminer le comportement asymptotique de $f(n)$.
\end{enumerate}
\end{exo}

\begin{exo} (Difficile.)
R\'esoudre la r\'ecurrence (discuter en fonction de $a_0$)
$$
a_n = a_{n-1}^2 + 2 \quad \forall n \geqslant 1
$$
(Hint~: poser $a_n = b_n + 1/b_n$.)
\end{exo}

\begin{exo} (Difficile.)
Montrer que la solution de la r\'ecurrence 
$$
\begin{array}{l}
a_n = \sin (a_{n-1}) \quad \forall n \geqslant 1\\
a_0 = 1
\end{array}
$$
v\'erifie $\lim_{n \to \infty} a_n = 0$ et $a_n = O(1/\sqrt{n})$.

(Hint~: poser $b_n = 1/a_{n}$.)
\end{exo} \newpage

\section{Séance 10 et 11}

\begin{exo}
Que vaut
$$
\sum_{n=0}^\infty H_n \frac{1}{10^n}\quad ?
$$
(Rappel~: $H_n$ est le $n$-\`eme nombre harmonique.)
\end{exo}

Nous savons que $$ C(x) = \sum_{n=0}^\infty H_n x^n = \frac{1}{1-x} ln(\frac{1}{1-x})$$

Donc $$ \sum_{n=0}^\infty H_n \frac{1}{10^n} = \frac{1}{1-\frac{1}{10}} ln(\frac{1}{1-\frac{1}{10}}) = \frac{10}{9} ln(\frac{10}{9}) = \frac{10}{9} (ln(10) - ln(9)) $$
%--------------------

\begin{exo}
Trouver la fonction g\'en\'eratrice ordinaire de $(2^n + 3^n)_{n \in \mathbb{N}}$, en forme close.
\end{exo}

%--------------------

\begin{exo} (Examen janvier 2011.) 
Calculer la somme de chacune des s\'eries suivantes.
%
\begin{enumerate}[a)]
\item $\displaystyle \sum_{n=0}^\infty \frac{H_n}{2^n}$
\item $\displaystyle \sum_{n=0}^\infty {n \choose 2} \frac{1}{10^n}$
\end{enumerate}
\end{exo}

\begin{enumerate}[a)]
\item Il s'agit de la FGO de $\frac{1}{1-\frac{1}{2}}*ln(\frac{1}{1-\frac{1}{2}}) = 2*ln(2)$.

\item Nous savons que 
\end{enumerate}

%--------------------

\begin{exo} (Examen ao\^ut 2011.) 
Calculer la somme de chacune des s\'eries suivantes.
%
\begin{enumerate}[a)]
\item $\displaystyle \sum_{n=1}^\infty \frac{1}{2^n}$
\item $\displaystyle \sum_{n=1}^\infty \frac{n}{2^n}$
\item $\displaystyle \sum_{n=1}^\infty \frac{1}{n 2^n}$ 
\end{enumerate}
\end{exo}

ATTENTION: n=1 dans chaque exercice!

\begin{enumerate}[a)]
\item $\displaystyle \sum_{n=1}^\infty \frac{1}{2^n} = \sum_{n=0}^\infty \frac{1}{2^n} - 1 = \frac{1}{1-\frac{1}{2}} - 1 = 2 - 1 = 1 $
\item $\displaystyle \sum_{n=1}^\infty \frac{n}{2^n}$
\item $\displaystyle \sum_{n=1}^\infty \frac{1}{n 2^n}$ 
\end{enumerate}

%--------------------

\begin{exo}
Quelle est la FGO de $(1,1+3,1+3+3^2,1+3+3^2+3^3,\ldots)$?
\end{exo}

%--------------------

\begin{exo} (Examen septembre 2015.)
R\'esolvez par la m\'ethode des fonctions g\'en\'eratrices l'\'equation de r\'ecurrence 
$$
a_n-3a_{n-1}=4^n \quad \mathrm{avec} \quad a_0=1.
$$
\end{exo}

%--------------------

\begin{exo}
Un collectionneur excentrique rafolle des pavages de rectangles $2 \times n$ par des dominos verticaux $2 \times 1$ et horizontaux $1 \times 2$. Il paye sans h\'esiter $4$\euro{} par domino vertical et $1$\euro{} par domino horizontal. Pour combien de pavages sera-t-il pr\^et \`a payer $n$\euro~?
\end{exo}

%--------------------\newpage % Le contenu
\end{document}