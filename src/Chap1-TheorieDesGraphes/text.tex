
\section{Théorie des Graphes}

\subsection{Définitions}

\begin{defn}
Un graphe $\Gamma$ est un triplet $(V,E,\gamma)$ où V est un ensemble fini dont les éléments sont appelés sommets, E est un ensemble fini dont les éléments sont appelés arêtes, $\gamma$ est une fonction $\gamma : E \rightarrow Paires(V)$. On nottera le plus souvent $\Gamma = (V,E)$ en omettant la fonction $\gamma$.\\

Soit $\gamma(e) = \{x,y\}$ pour $e \in E, x,y \in V$:
\begin{center}
	\begin{enumerate}
	\item On dit que x et y sont adjacents.

	\item On dit que e est incidente à x et y. \\
	\end{enumerate}
\end{center}

\end{defn}


\begin{defn}
Soit $\Gamma = (V,E,\gamma)$ un graphe.

\begin{enumerate}
\item $\gamma(e)= \{x,x\}$ pour $e \in E, x \in V$ est appellé un lacet.
\item Si au moins 2 arêtes sont incidentes à 2 mêmes somments, on les appelle arêtes multiples.
\item Un graphe est simple s'il n'a ni lacet, ni arêtes multiples. Dans ce cas, on omet la fonction $\gamma$,on note $\Gamma = (V,E)$ et E est identifié un sous-ensemble de Paires(V). \\
\end{enumerate}
\end{defn}

\begin{defn}
Soit $\Gamma = (V,E)$ un graphe. Le degré d'un sommet $v \in V$ est le nombre d'arêtes incidentes à v, les lacets comptant pour 2 arêtes. On note le degré de v par deg(V).
\end{defn}

\begin{exmp}
Dans la figure suivante, nous avons 2 sommets de degré 4 et 6 sommets de degré 1.
\end{exmp}

\begin{figure}[htb]
	\centering
	\begin{tikzpicture}[>=stealth',shorten >=1pt,auto,node distance=1.5cm,thick,main node/.style={circle,fill=blue!20,draw,font=\sffamily\large\bfseries}]

	\node[main node] (c1) {C};
	\node[main node] (c2) [right of=c1] {C};
	\node[main node] (h1) [above of=c1] {H};
	\node[main node] (h2) [left of=c1] {H};
	\node[main node] (h3) [below of=c1] {H};
	\node[main node] (h4) [above of=c2] {H};
	\node[main node] (h5) [right of=c2] {H};
	\node[main node] (h6) [below of=c2] {H};

	\path[every node/.style={font=\sffamily\small}]
	(c1) edge node [above] {} (h1)
		 edge node [left] {} (h2)
		 edge node [below] {} (h3)
		 edge node [right] {} (c2)
	(c2) edge node [above] {} (h4)
		 edge node [right] {} (h5)
		 edge node [below] {} (h6) ;

	\end{tikzpicture}

	\caption{Exemple degrés des sommets dans la molécule $C_{2}H_{6}$.}
\end{figure}

\begin{thrm}
Soit $\Gamma = (V,E)$, alors $$\sum_{i=1}^{\#V} deg(v_{i}) = 2\#E$$
\end{thrm}

\begin{demo}
Chaque arête contribue 2 fois dans la somme des degrés.
\end{demo}

\begin{corll}
La somme des degrés des sommets d'un graphe est paire. \\
\end{corll}

\begin{defn}
Le graphe complet $K_{n}$ est le graphe simple à n sommets pour lequel chaque paire de sommets est une arête.
\end{defn}

\begin{exmp}
<Dessin des graphes complets $K_1 à K_5$>\\
\end{exmp}

\begin{defn}
Un graphe ${\Gamma}'=(U,F)$ est un sous-graphe de $\Gamma=(V,E)$ si $ U \subseteq V$ et $F \subseteq E$. On nottera $ {\Gamma}' \leq \Gamma$.
\end{defn}

\begin{exmp}
$ K_{m} \leq K_{n}$ si $ m \leq n$.
\end{exmp}

\begin{exo}
Montrer que $K_{m}$ possède $ q=\frac{1}{2}n(n-1)$ arêtes.
\end{exo}

\subsection{Chemins dans les graphes}

\begin{defn}

\end{defn}

\begin{defn}

\end{defn}

\begin{defn}

\end{defn}

\subsection{Arbres}

\begin{defn}

\end{defn}

\begin{defn}

\end{defn}

\begin{exmp}

\end{exmp}

\begin{prop}

\end{prop}

\begin{demo}

\end{demo}

\begin{thrm}

\end{thrm}

\begin{demo}

\end{demo}